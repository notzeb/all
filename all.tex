\documentclass[letterpaper,11pt]{report}
\usepackage{amsfonts,amssymb,amsmath,amsthm,latexsym}
\usepackage{stmaryrd}
\usepackage[all]{xy}
\usepackage{fullpage}
\usepackage{hyperref}
\usepackage{enumitem,todonotes}

\usepackage{algorithm}
\usepackage[noend]{algpseudocode}
\usepackage{graphicx}
%\usepackage{fullpage}

% from stack exchange
\usepackage{sistyle}
\SIthousandsep{,}

\DeclareMathOperator\Supp{Supp}
\DeclareMathOperator\Tr{Tr}

\DeclareMathOperator{\Clo}{Clo}
\DeclareMathOperator{\Inv}{Inv}
\DeclareMathOperator{\lcm}{lcm}
\DeclareMathOperator{\CSP}{CSP}
\DeclareMathOperator{\Sg}{Sg}
\DeclareMathOperator{\Aut}{Aut}

\DeclareMathOperator{\rad}{rad}
\DeclareMathOperator{\diam}{diam}
\DeclareMathOperator{\inter}{int}
\DeclareMathOperator{\supp}{supp}

\begin{document}

\makeatletter
\newtheorem*{rep@theorem}{\rep@title}
\newcommand{\newreptheorem}[2]{%
\newenvironment{rep#1}[1]{%
 \def\rep@title{#2 \ref{##1}}%
 \begin{rep@theorem}}%
 {\end{rep@theorem}}}
\makeatother

\newtheorem{thm}{Theorem}
\newreptheorem{thm}{Theorem}
\newtheorem{prop}{Proposition}
\newtheorem{cor}{Corollary}
\newtheorem{lem}{Lemma}
\newreptheorem{lem}{Lemma}

\theoremstyle{definition}
\newtheorem{defn}{Definition}
\newtheorem{conj}{Conjecture}
\newtheorem{prob}{Problem}

\theoremstyle{remark}
\newtheorem{rem}{Remark}
\newtheorem{ex}{Example}
\newtheorem{exer}{Exercise}

\newcommand{\Rho}{\mathrm{P}}
\newcommand{\cS}{\mathcal{S}}
\newcommand{\cM}{\mathcal{M}}
\newcommand{\cN}{\mathcal{N}}
\newcommand{\gk}{\kappa}
\newcommand{\gS}{\Sigma}
\newcommand{\gl}{\lambda}
\newcommand{\gt}{\theta}

\newcommand{\cF}{\mathcal{F}}
\newcommand{\cG}{\mathcal{G}}
\newcommand{\cP}{\mathcal{P}}
\newcommand{\cV}{\mathcal{V}}
\newcommand{\cB}{\mathcal{B}}
\newcommand{\cA}{\mathcal{A}}
\newcommand{\ZZ}{\mathbb{Z}}
\newcommand{\NN}{\mathbb{N}}
\newcommand{\QQ}{\mathbb{Q}}
\newcommand{\bA}{\mathbb{A}}
\newcommand{\bB}{\mathbb{B}}
\newcommand{\bC}{\mathbb{C}}
\newcommand{\bD}{\mathbb{D}}
\newcommand{\bE}{\mathbb{E}}
\newcommand{\bF}{\mathbb{F}}
\newcommand{\bI}{\mathbb{I}}
\newcommand{\bP}{\mathbb{P}}
\newcommand{\bS}{\mathbb{S}}
\newcommand{\fA}{\mathbf{A}}
\newcommand{\fB}{\mathbf{B}}

\newcommand{\RR}{\mathbb{R}}
\newcommand{\CC}{\mathbb{C}}
\newcommand{\FF}{\mathbb{F}}
\newcommand{\HH}{\mathbb{H}}
\newcommand{\PP}{\mathbb{P}}
\newcommand{\EE}{\mathbb{E}}

\newcommand{\cD}{\mathcal{D}}
\newcommand{\cK}{\mathcal{K}}
\newcommand{\cL}{\mathcal{L}}
\newcommand{\cO}{\mathcal{O}}

\newcommand{\fp}{\mathfrak{p}}

\newcommand{\dotcup}{\ensuremath{\mathaccent\cdot\cup}}

\title{Notes}
\date{}
\author{}
\maketitle

\tableofcontents


\chapter{Algebra}

\section{Noncommutative rings}

\begin{defn} If $R$ is a ring, then the \emph{Jacobson radical} $J(R)$ (sometimes written $\rad(R)$) is the intersection of the annihilators of all simple left $R$-modules.
\end{defn}

\begin{defn} A submodule $N$ of $M$ is \emph{superfluous}, written $N \subseteq_s M$ or $N \ll M$, if for all $H$ we have $N+H = M\ \implies\ H = M$.
\end{defn}

\begin{thm} We can replace ``left'' by ``right'' in the definition of the Jacobson radical of a ring. Furthermore, we have the following equivalent definitions:
\begin{itemize}
\item $J(R)$ is the intersection of all maximal left ideals of $R$,
\item $J(R)$ is the sum of all superfluous left ideals of $R$,
\item $J(R)$ is the maximal left ideal of $R$ such that for all $x \in J(R)$, $1-x$ has a left inverse,
\item $J(R) = \{x \in R \mid 1+RxR \subseteq R^\times\}$.
\end{itemize}
\end{thm}

\begin{lem}[Nakayama's Lemma] If $M$ is a finitely generated left $R$-module with $M = J(R)M$, then $M=0$.
\end{lem}
\begin{proof} Consider a minimal generating set $x_1, ..., x_n$ of $M$, and use $\sum x_i \in J(R)M$ to write $x_n$ as a linear combination of $x_1, ..., x_{n-1}$.
\end{proof}

\begin{prop} $J(R/J(R)) = 0$.
\end{prop}

\subsection{Artinian Rings}

\begin{prop} If $R$, considered as a left $R$-module over itself, has a composition series of length $k$, then $J(R)^k = 0$.
\end{prop}

\begin{thm}[Hopkins' Theorem] If $M$ is a left module over a left Artinian ring, then the following are equivalent:
\begin{itemize}
\item $M$ is finitely generated,
\item $M$ has finite length,
\item $M$ is Noetherian,
\item $M$ is Artinian.
\end{itemize}
\end{thm}

\begin{thm}[Hopkins-Levitzki] If $R$ is \emph{semiprimary} - that is, if $R/J(R)$ is semisimple and $J(R)$ is nilpotent - then for left $R$-modules, being Noetherian, being Artinian, and having a composition series are equivalent.
\end{thm}

\begin{prop} If $J(R) = 0$, then every minimal left ideal of $R$ is a direct summand of $R$.
\end{prop}

\begin{thm} $R$ is semisimple if and only if it is left Artinian and has $J(R) = 0$.
\end{thm}

% TODO: Schur-Weyl, maybe via https://mathoverflow.net/questions/255492/how-to-constructively-combinatorially-prove-schur-weyl-duality





\section{Commutative Algebra}

\begin{defn} If $R$ is a commutative ring, then $I \lhd R$ means that $I$ is an ideal of $R$.
\end{defn}

\begin{defn} If $I,J \lhd R$, set $(I:J) = \{r \in R \mid rJ \subseteq I\}$. If $a \in R$, we abbreviate $(I:(a))$ to $(I:a)$.
\end{defn}

\subsection{Primary Ideals}

\begin{defn} $Q \lhd R$ is \emph{primary} if $\forall a,b\in R$ with $ab \in Q$, either $b \in Q$ or $\exists n$ such that $a^n \in Q$.
\end{defn}

\begin{defn} If $I \lhd R$, then $\rad(I) = \{r \in R \mid \exists n\ r^n \in I\}$.
\end{defn}

\begin{prop} $Q$ is primary if and only if $\rad(Q)$ is prime. If $Q_1, Q_2$ are primary and $\rad(Q_1) = \rad(Q_2)$, then $Q_1 \cap Q_2$ is primary. If $R$ is Noetherian and $Q \lhd R$, then $\exists n$ such that $\rad(Q)^n \subseteq Q$.
\end{prop}

\begin{thm}[Primary Decomposition] If $R$ is Noetherian and $I \lhd R$, then $\exists k$ and $Q_1, ..., Q_k \lhd R$ primary such that $I = Q_1 \cap \cdots \cap Q_k$.
\end{thm}
\begin{proof} By $R$ Noetherian, $\forall a\in R\ \exists n$ with $(I:a^n) = (I:a^{n+1})$, and for this $n$ we have $(I+(a^n))\cap (I:a) = I$, so either $I$ is already primary or we can write $I$ as an intersection of bigger ideals, and apply Noetherian induction.
\end{proof}

\begin{lem} If $R$ is Noetherian, then for any $I \lhd R$ and $r \in R \setminus I$, there exists $s \in R$ such that $(I:rs)$ is prime.
\end{lem}

\begin{thm}[Uniqueness of radicals] If $R$ is Noetherian, $I = Q_1 \cap \cdots \cap Q_k$ with $Q_i \lhd R$ primary and no $Q_i$ containing $\cap_{j \ne i} Q_j$, and if $\fp \lhd R$ is prime, then $\exists r \in R$ with $(I:r) = \fp$ if and only if there is an $i$ with $\rad(Q_i) = \fp$. In particular, the set $\{\rad(Q_i)\}_{i \le k}$ is uniquely determined by $I$.
\end{thm}

\begin{thm}[Uniqueness of primaries with minimal radical] If $R$ is Noetherian, $I = Q_1 \cap \cdots \cap Q_k$ with $Q_i \lhd R$ primary and $\rad(Q_i) \not\subseteq \rad(Q_1)$ for $i > 1$, then for $n$ sufficiently large we have $(I:\rad(Q_2)^n \cdots \rad(Q_k)^n) = Q_1$, so $Q_1$ is uniquely determined by $I$ and $\rad(Q_1)$.
\end{thm}

% TODO: Group theory! Nielsen reduction, Frobenius groups, Frattini subgroups, Frattini's argument, primitive groups, CFSG, ...


\chapter{Analysis}

\section{Basic Facts}

\subsection{Point Set Stuff}

\begin{defn} A topological space is \emph{normal}, or $T_4$, if any two disjoint closed sets have disjoint open neighborhoods.
\end{defn}

\begin{prop} Compact Hausdorff spaces are normal.
\end{prop}

\begin{lem}[Urysohn's Lemma]\label{urysohn} A topological space $X$ is normal iff for any disjoint closed subsets $A,B \subseteq X$ there exists a continuous $f : X \rightarrow [0,1]$ such that $f(A) \subseteq \{0\}$ and $f(B) \subseteq \{1\}$.
\end{lem}
\begin{proof} Let $U(1) = X\setminus B, V(0) = X\setminus A$. For each dyadic rational $r = \frac{2a+1}{2^{n+1}} \in (0,1)$ we construct disjoint open subsets $U(r), V(r) \subseteq X$ such that $X \setminus V(\frac{a}{2^n}) \subseteq U(\frac{2a+1}{2^{n+1}})$ and $X\setminus U(\frac{a+1}{2^n}) \subseteq V(\frac{2a+1}{2^{n+1}})$. Then for every $r$ we have $A \subseteq U(r)$, $B \subseteq V(r)$, for $r \le s$ we have $U(r) \cap V(s) = \emptyset$, and for $r < s$ we have $V(r) \cup U(s) = X$. Thus for $r < s$, the closure of $U(r)$ is contained in $U(s)$. Finally, define $f$ by $f(x) = \min(1,\inf\{r \mid x \in U(r)\})$.
\end{proof}

\begin{lem}[Locally Compact Urysohn's Lemma]\label{lch-urysohn} If $X$ is locally compact Hausdorff and $K \subseteq U \subseteq X$ with $K$ compact and $U$ open, then there exists a continuous $f : X \rightarrow [0,1]$ such that $f(K) \subseteq \{1\}$ and $f$ is supported on a compact subset of $U$.
\end{lem}
\begin{proof} Find a precompact open set $V$ with $K \subseteq V \subseteq \overline{V} \subseteq U$, then $\overline{V}$ is normal (since it is compact and Hausdorff), so by Urysohn's Lemma there is a continuous $f : \overline{V} \rightarrow [0,1]$ with $f(K) \subseteq \{1\}$ and $f(\partial V) \subseteq \{0\}$.
\end{proof}

\begin{thm}[Tietze Extension Theorem]\label{tietze} If $X$ is a normal space, $A \subseteq X$ is closed, and $f : A \rightarrow \RR$ is continuous, then there exists a continuous $F : X \rightarrow \RR$ with $F|_A = f$.
\end{thm}
\begin{proof} Assume without loss of generality that $f(A) \subseteq [0,1]$. We'll find a sequence of functions $g_i: X \rightarrow [0, \frac{2^{i-1}}{3^i}]$ with $0 \le f - \sum_{i=1}^n g_i \le \frac{2^n}{3^n}$ for all $n$, and finish by taking $F = \sum_i g_i$. It's enough to show how to find $g_1$: we apply Urysohn's Lemma to find $g_1 : X \rightarrow [0,\frac{1}{3}]$ with $g_1(x) = 0$ for $x \in f^{-1}([0,\frac{1}{3}])$ and $g_1(x) = \frac{1}{3}$ for $x \in f^{-1}([\frac{2}{3},1])$.
\end{proof}

\begin{cor}[Locally Compact Tietze]\label{lch-tietze} If $X$ is locally compact Hausdorff and $K \subseteq U \subseteq X$ with $K$ compact and $U$ open, then for every continuous $f : K \rightarrow [0,1]$ there exists a continuous $F : X \rightarrow [0,1]$ such that $F|_K = f$ and $F$ is supported on a compact subset of $U$.
\end{cor}

\begin{prop} If $X$ is locally compact Hausdorff, then $C_0(X)$ is the closure of $C_c(X)$ in the uniform metric.
\end{prop}

\begin{thm}[Stone-Weierstrauss, lattice version] If $X$ compact, $B \subseteq C(X,\RR)$ such that for any $x,y \in X$ and $a,b \in \RR$ there exists $g \in B$ with $g(x) = a, g(y) = b$, and such that $B$ contains $\max(f,g), \min(f,g)$ whenever it contains $f,g$, then $B$ is dense in $C(X,\RR)$.
\end{thm}
\begin{proof} Let $f \in C(X,\RR)$, and for all $x,y \in X$ pick $g_{xy} \in B$ with $g_{xy}(x) = f(x), g_{xy}(y) = f(y)$. Fix $\epsilon > 0$. Take $U_{xy} = \{z \mid f(z) < g_{xy}(z) + \epsilon\}, V_{xy} = \{z \mid f(z) > g_{xy} - \epsilon\}$. For any $y$, some finite subcollection of the $U_{xy}$s cover $X$, corresponding to $x_1, ..., x_n$, take $g_y = \max(g_{x_1y}, ..., g_{x_ny})$ and $V_y = \cap V_{x_jy}$, then $f < g_y + \epsilon$ and for $x \in V_y$ we have $f(x) > g_y(x) - \epsilon$. Now take a finite subcollection of the $V_y$s which covers $X$, and let $g$ be the minimum of the corresponding $g_y$s, then $g \in B$ and $|f-g| \le \epsilon$.
\end{proof}

\begin{defn} The \emph{Bernstein polynomials} are defined by
\[
b_{\nu,n}(x) = \binom{n}{\nu} x^\nu (1-x)^{n-\nu}.
\]
\end{defn}

\begin{thm}[Weierstrauss approximation] If $f:[a,b]\rightarrow \CC$ is continuous, then $\forall \epsilon > 0$ there exists a polynomial $p \in \CC[x]$ such that $\forall x \in [a,b]$, we have $|f(x) - p(x)| < \epsilon$.
\end{thm}
\begin{proof} Suppose $[a,b] = [0,1]$, and define $B_n(f)$ by
\[
B_n(f) = \sum_{\nu = 0}^n f(\tfrac{\nu}{n})b_{\nu,n}.
\]
If $k$ is the number of times we flip heads in $n$ independent random coinflips with bias $x$, then
\[
\EE[f(\tfrac{k}{n})] = B_n(f)(x),
\]
so the law of large numbers shows that $B_n(f)$ approximates $f$.
\end{proof}

\begin{thm}[Stone-Weierstrauss for $\RR$] $X$ compact Hausdorff, $A$ a subalgebra of $C(X,\RR)$ which contains a non-zero constant. Then $A$ is dense in $C(X,\RR)$ iff it separates points.
\end{thm}
\begin{proof} It's enough to show that if $f \in A$, then $|f|$ is in the closure of $A$, since then the closure of $A$ will be closed under $\max$ and $\min$. To do this, we find $p \in \RR[x]$ such that $\forall x \in f(X)$ we have $||x| - p(x)| < \epsilon$, then $p\circ f \in A$ and $||f| - p\circ f| < \epsilon$.
\end{proof}

\begin{thm}[Stone-Weierstrauss for $\CC$] $X$ compact Hausdorff, $S \subseteq C(X,\CC)$ separates points. Then the complex unital $*$-algebra generated by $S$ is dense in $C(X,\CC)$.
\end{thm}

\begin{thm}[Tychonoff's Theorem] If $\{X_a\}_{a \in A}$ is a family of compact sets, then $X = \prod_{a \in A} X_a$ is compact in the product topology.
\end{thm}
\begin{proof} With nets and Zorn: Suppose $\{U_i\}_{i \in I}$ is an open cover of $X$ with no finite subcover. For each finite subset $J \subseteq I$, let $x_J$ be a point of $X$ not contained in $\cup_{j \in J} U_j$. We show that for every $B \subseteq A$, the net $\{\pi_B(x_J)\}_{J \subseteq I}$ has a cluster point, by transfinite induction on $B$, and taking $B = A$ gives a contradiction.

With Alexander Subbase Theorem \ref{alexander-subbase}: Suppose there is an open cover by cylinder sets with no finite subcover. Then for each $a \in A$, there is some $x_a \in X_a$ not covered by the cylinder sets corresponding to coordinate $a$, and the corresponding point $(x_a)_{a \in A} \in X$ is not covered by any of the cylinders.
\end{proof}

\begin{thm}[Alexander Subbase Theorem]\label{alexander-subbase} If $X$ is a topological space with subbase $B$, and every open cover of $X$ by elements of $B$ has a finite subcover, then $X$ is compact.
\end{thm}
\begin{proof} Suppose not, let $C$ be a maximal open cover of $X$ which has no finite subcover (alternatively, take $C$ to be a maximal proper ideal of $\cP(X)$ containing an open cover of $X$). Take $x \in X$ not contained in any element of $C\cap B$, then there is $U \in C$ with $x \in U$, and $S_1, ..., S_n \in B$ with $x \in S_1 \cap \cdots \cap S_n \subseteq U$. For each $S_i$, since $S_i \not\in C$ there must be a finite subset $C_i \subseteq C$ such that $\{S_i\} \cup C_i$ covers $X$, but then $\{U\} \cup C_1 \cup \cdots \cup C_n$ is a finite subset of $C$ which covers $X$.
\end{proof}

The next bit is from \url{https://math.stackexchange.com/a/6338}.

\begin{defn} A topological space is called a \emph{continuum} if it is a compact connected Hausdorff space.
\end{defn}

\begin{lem} Let $X$ be a continuum. If $F$ is a non-trivial closed subset of $X$, then for every component $C$ of $F$ we have that $\partial F \cap C$ is non-empty.
\end{lem}
\begin{proof} Since $X$ is Hausdorff compact, quasicomponents coincide with components, so $C$ is the intersection of all clopen sets in $F$ which contain $C$. Suppose that $C$ is disjoint from $\partial F$. Then, by compactness of $\partial F$, there is a single clopen set $A$ in $F$ containing $C$ and disjoint from $\partial F$. Take an open set $U$ such that $A = U \cap F$. $A \cap \partial F = \emptyset$ implies that $A = U \cap \inter(F)$, so $A$ is open in $X$. But $A$ is also closed in $X$, and contains $C$, so $A=X$. But then $\partial F = \emptyset$, which is not possible since $F$ would be non-trivial clopen in $X$.
\end{proof}

\begin{thm}[Sierpi\'nski \cite{sierpinski-continuum}] If a continuum $X$ has a countable cover $\{X_i\}_{i=1}^{\infty}$ by pairwise disjoint closed subsets, then at most one of the sets $X_i$ is non-empty.
\end{thm}
\begin{proof} Assume that at least two of the sets $X_i$ are non-empty. First we show that for every $i$ there exists a continuum $C \subseteq X$ such that $ C \cap X_i = \emptyset$ and at least two sets in the sequence $C \cap X_1, C \cap X_2, \ldots$ are non-empty. If $X_i$ is empty then we can take $C = X$; thus we can assume that $X_i$ is non-empty. Take $j \ne i$ such that $X_j \ne \emptyset$. Since $X$ is Hausdorff compact, there are disjoint open sets $U,V \subseteq X$ satisfying $X_i \subseteq U$ and $X_j \subseteq V$. Let $C$ be a component of $\overline{V}$ which meets $X_j$. Clearly, $C$ is a continuum, $C \cap X_i = \emptyset$ and $C \cap X_j \ne \emptyset$. By the previous lemma, $C \cap \partial (\overline{V}) \ne \emptyset$ and since $X_j \subseteq \inter(\overline{V})$, there exist a $k \ne j$ such that $C \cap X_k \ne \emptyset$.

It follows that there exists a decreasing sequence $C_1 \supseteq C_2 \ \supseteq \ldots$ of continua contained in $X$ such that $C_i \cap X_i = \emptyset$ and $C_i \ne \emptyset$ for $i=1,2, \ldots$ The first part implies that $\bigcap_{i=1}^{\infty} C_i = \emptyset$ and from the second part and compactness of $X$ it follows that $\bigcap_{i=1}^{\infty} C_i \ne \emptyset$.
\end{proof}

\begin{defn} A subset $S$ of a topological space is \emph{perfect} if it is closed and every point of $S$ is a limit point.
\end{defn}

\begin{defn} A \emph{Polish space} is a separable completely metrizable topological space.
\end{defn}

\begin{thm}[Cantor] Every nonempty perfect subset of a Polish space has cardinality at least $2^{\aleph_0}$.
\end{thm}

\begin{defn} A \emph{condensation point} of a subset $S$ of a topological space is a point $x$ such that every neighborhood of $x$ intersects $S$ in uncountably many points.
\end{defn}

\begin{thm}[Cantor-Bendixson] Every closed subset $S$ of a Polish space $X$ can be written uniquely as a disjoint union of a perfect set and a countable set.
\end{thm}
\begin{proof} Ordinal proof: For any set $S$, let $S'$ be the set of limit points of $S$. Define a sequence $S_\alpha$ indexed by ordinals by $S_0 = S$, $S_{\alpha+1} = S_\alpha'$, and $S_\beta = \cap_{\alpha < \beta} S_\alpha$ for $\beta$ a limit ordinal. Since each closed set $S_\alpha$ is determined by the collection of open subsets of a basis of $X$ which do not intersect it, and since every well-ordered chain contained in $\mathcal{P}(\mathbb{N})$ is countable, there is some countable ordinal $\beta$ such that $S_\beta = S_{\beta+1}$. Since the number of isolated points of any $S_\alpha$ is countable, we see that $S\setminus S_\beta$ must be countable.

Condensation point proof: Let $P$ be the set of condensation points of $S$. Then $S\setminus P$ is contained in a countable union of open sets of a basis of $X$ which each intersect $S$ in countably many points, so $S\setminus P$ is countable and $P$ is perfect.

For uniqueness: note that every point in a perfect subset of $S$ must be a condensation point of $S$.
\end{proof}

\subsection{Metric Spaces}

\begin{defn} A metric space is \emph{complete} if every Cauchy sequence has a limit. It is \emph{totally bounded} if it can be covered by finitely many subsets of size $\epsilon$, for every $\epsilon > 0$.
\end{defn}

\begin{thm} A metric space is compact iff it is complete and totally bounded.
\end{thm}

\begin{defn} A metric space is \emph{sequentially compact} if every sequence has a bounded subsequence.
\end{defn}

\begin{thm}[Bolzano-Weierstrauss] A subset of $\RR^n$ is sequentially compact iff it is closed and bounded.
\end{thm}

\begin{prop} A closed subset of a complete space is complete, and a complete subset of a metric space is closed.
\end{prop}

\begin{thm}[Baire Category Theorem] If $M$ is either a complete metric space or a locally compact Hausdorff space, then a union of countably many nowhere dense subsets of $M$ has empty interior.
\end{thm}

\begin{defn} A space is called a \emph{Baire space} if the intersection of any countable collection of open dense sets is dense.
\end{defn}

\begin{thm}[Banach Fixed Point] Contraction mappings on complete metric spaces have unique fixed points.
\end{thm}

\begin{cor}[Picard-Lindel\"of] The initial value problem $y'(t) = f(t,y(t)), y(t_0) = y_0$ for $t \in [t_0-\epsilon,t_0+\epsilon]$ has a unique solution for some $\epsilon > 0$ if $f$ is Lipschitz continuous in $y$ and continuous in $t$.
\end{cor}

\begin{defn} If $X,Y$ are Banach spaces, $U \subseteq X$ open, then $f:U\rightarrow Y$ is called \emph{Frech\'et differentiable} at $x$ if there exists a bounded linear operator $A:X\rightarrow Y$ such that $\|f(x+h) - f(x) - Ah\|_Y = o(\|h\|_X)$ as $h \rightarrow 0$. In this case we write $Df_x = A$.
\end{defn}

\begin{cor}[Inverse Function Theorem] If $X,Y$ are Banach spaces, $U$ an open neighborhood of $0$ in $X$, $F:U\rightarrow Y$ continuously (Fr\'echet) differentiable and $DF_0:X\rightarrow Y$ a bounded isomorphism from $X$ to $Y$ (with bounded inverse), then there exists an open neighborhood $V \subseteq Y$ of $F(0)$ and a continuously differentiable map $G:V\rightarrow X$ such that $F(G(y)) = y$ for all $y \in V$.
\end{cor}

\begin{defn} A topological space is called \emph{separable} if it contains a countable dense set. It is called \emph{second countable} if its topology has a countable base.
\end{defn}

\begin{prop} Every second countable space is separable, and every separable metric space is second countable.
\end{prop}

\begin{defn} If $X,Y$ are metric spaces, then $f:X \rightarrow Y$ is called \emph{uniformly continuous} if $\forall \epsilon > 0\ \exists \delta > 0$ such that $\forall x,y \in X$ such that $d_X(x,y) < \delta$, we have $d_Y(f(x),f(y)) < \epsilon$.
\end{defn}

\begin{defn} A family of functions $F$ is called \emph{equicontinuous} at $x_0 \in X$ if $\forall \epsilon > 0\ \exists \delta > 0$ such that $\forall f \in F, x \in X$ such that $d(x_0,x) < \delta$ we have $d(f(x_0),f(x)) < \epsilon$. $F$ is \emph{uniformly equicontinuous} if $\forall \epsilon > 0\ \exists \delta > 0$ such that $\forall f \in F, x,y$ such that $d(x,y) < \delta$ we have $d(f(x),f(y)) < \epsilon$.
\end{defn}

\begin{thm}[Arzel\`a-Ascoli] If $(f_n)_{n \in \NN}$ defined on $[a,b]$ is uniformly bounded and equicontinuous, then there is a subsequence which converges uniformly.
\end{thm}

\begin{thm}[Ascoli Version 2] If $X$ is compact Hausdorff, then a subset of $C(X)$ (with the uniform norm) is compact iff it is closed, pointwise bounded, and equicontinuous.
\end{thm}

\begin{lem}[Finite Vitali Covering Lemma]\label{finite-vitali} If $B_1, ..., B_n$ are balls in a metric space, then there is a subcollection $B_{j_1}, ..., B_{j_k}$ which are disjoint, and which satisfy
\[
B_1 \cup \cdots \cup B_n \subseteq 3B_{j_1} \cup \cdots \cup 3B_{j_k},
\]
where $3B_j$ is the ball with the same center as $B_j$ and three times the radius.
\end{lem}
\begin{proof} Keep adding the biggest ball which is disjoint from the ones you have chosen so far to your collection. Then every ball you haven't chosen will intersect a larger ball that you have chosen.
\end{proof}

\begin{lem}[Infinite Vitali Covering Lemma]\label{vitali} If $(B_i)_{i\in I}$ is a collection of balls in a metric space such that $\sup_{i\in I} \rad(B_i) < \infty$, then for any $c > 1$ there is a subcollection $J \subseteq I$ such that the $B_j$ with $j \in J$ are disjoint, and $\cup_{i\in I} B_i \subseteq \cup_{j\in J} (1+2c)B_j$.
\end{lem}
\begin{proof} Let $R = \sup \rad(B_i)$, and for each $n$ choose a maximal disjoint subcollection of the balls with radius between $R/c^n$ and $R/c^{n+1}$ which are disjoint from the balls you have already chosen so far. Then every ball you haven't chosen will intersect a ball you have chosen, whose radius is at most a factor of $c$ smaller.
\end{proof}

\begin{lem}[Besicovitch Covering Lemma]\label{besicovitch} For every $n$ there exists a constant $c_n$ such that for $E \in \mathbb{R}^n$ bounded and for a collection of balls $\mathcal{B}$ such that every point of $E$ is the center of some ball $B$ in $\mathcal{B}$, there is a collection of $c_n$ families $\mathcal{B}_i \subseteq \mathcal{B}$ of pairwise disjoint balls, such that $E \subseteq \cup_{i \le c_n} \cup_{B \in \mathcal{B}_i} B$.
\end{lem}
\begin{proof} WLOG assume all balls in $\mathcal{B}$ are contained in a big ball $B_0$. Make a countable sequence of balls $B_i \in \mathcal{B}$ such that each $B_i$ has its center not contained in $B_1, ..., B_{i-1}$, and its radius within a factor of $1-\epsilon$ of the $\sup$ of the radii of such balls. If we shrink each $B_i$ by a factor of $1+\frac{1}{1-\epsilon}$ to make a ball $B_i'$, the $B_i'$s are pairwise disjoint. Since the volume of $B_0$ is at least the sum of the volumes of the $B_i'$s, the radii of the $B_i$s goes to zero, so $E \subseteq \cup_i B_i$.

To finish, we just need to show that each $B_i$ intersects less than $c_n$ of the balls $B_1, ..., B_{i-1}$. To do this, we divide the balls $B_1, ..., B_{i-1}$ which intersect $B_i$ into two groups based on whether their radii are at most $M$ times the radius of $B_i$. The group of smaller balls is bounded because the $B_j'$s are disjoint and contained in a ball of radius $2M+1$ times the radius of $B_i$ (and the radii of the $B_j$ with $j < i$ are at least $1-\epsilon$ times the radius of $B_i$). The group of larger balls is bounded by showing that the angles between the rays connecting the center of $B_i$ with the centers of the $B_j$s must be large (approaching $\frac{\pi}{3}$) if $M$ is big enough and $\epsilon$ small enough (using the law of cosines).
\end{proof}


\subsection{Topologies on $C(X,Y)$}

\begin{defn} The \emph{compact-open} topology on $C(X,Y)$ has a subbase given by
\[
V(K,U) = \{f:X\rightarrow Y \mid f(K) \subseteq U\}
\]
for $K$ compact and $U$ open.
\end{defn}

\begin{prop} If $Y$ is a metric space then $f_n \rightarrow f$ in the compact-open topology iff $\forall K \subseteq X$ compact we have $f_n \rightarrow f$ uniformly on $K$, so in this case the compact-open topology is the ``topology of compact convergence''. If $X$ is compact as well, this becomes the uniform convergence topology.
\end{prop}

\begin{prop} If $Y$ is locally compact Hausdorff, composition $\circ : C(Y,Z) \times C(X,Y) \rightarrow C(X,Z)$ is continuous in the compact-open topology.
\end{prop}

\begin{defn} If $X,Y$ Banach spaces, $U\subseteq X$ open, $\mathcal{C}^m(U,Y)$ the $m$-times continuously Frech\'et-differentiable functions $U \rightarrow Y$, then the ``compact-open'' topology on $\mathcal{C}^m(U,Y)$ is induced by the seminorms
\[
\rho_K(f) = \sup\{\|D^jf_x\| \mid x \in K,\ 0 \le j \le m\}
\]
for $K \subseteq U$ compact.
\end{defn}

\begin{defn} The topology of \emph{compact convergence} is defined by $f_n \rightarrow f$ iff for all $K$ compact, $f_n|_K \rightarrow f|_K$ converges uniformly.
\end{defn}

\begin{prop} A set $F$ of functions is called \emph{normal} if every sequence of functions from $F$ contains a subsequence that converges compactly to a continuous function.
\end{prop}

\begin{thm}[Montel] Any uniformly bounded family of holomorphic functions defined on an open subset of $\CC$ is normal.
\end{thm}

\begin{defn} The topology of \emph{pointwise convergence} is the product topology on $Y^X$ - this has $f_n \rightarrow f$ iff $f_n(x) \rightarrow f(x)$ for all $x$.
\end{defn}


\subsection{Measure}

\begin{defn} Two subsets $A,B$ of $\RR^n$ are called \emph{equidecomposable} if $A$ can be cut into finitely many disjoint pieces which can be translated and rotated to give a disjoint decomposition of $B$. More generally, if $G$ is a group acting on a set $X$, then two subsets $A,B$ of $X$ are $G$-\emph{equidecomposable} if we can write $A = A_1 \dotcup \cdots \dotcup A_n$ and if there are $g_i \in G$ with $B = g_1A_1 \dotcup \cdots \dotcup g_nA_n$.% We say two sets are \emph{countably} $G$-\emph{equidecomposable} if we can do the same with countably many disjoint pieces.
\end{defn}

\begin{prop}[Banach-Cantor-Schr\"oder-Bernstein] If $A$ is equidecomposable with a subset of B and $B$ is equidecomposable with a subset of $A$, then $A,B$ are equidecomposable.
\end{prop}

\begin{defn} If $G$ acts on $X$ and $Y \subseteq X$, we say that $Y$ is $G$-\emph{paradoxical} if there are disjoint $A,B \subseteq Y$ which are both $G$-equidecomposable with $Y$.
\end{defn}

\begin{prop} If $F_2$ is the free group on two generators $a,b$, then we can write $F = A_0 \dotcup A_1 \dotcup A_2 \dotcup B_1 \dotcup B_2$, with $F = A_0 \dotcup aA_1 \dotcup A_2 = bB_1 \dotcup B_2$. In particular, $F_2$ is $F_2$-paradoxical.
\end{prop}
\begin{proof} For any word $w$, let $W(w)$ be the set of elements of $F_2$ that begin with $w$. Take $A_0 = \{a^{-n} \mid n \ge 0\}, A_1 = W(a^{-1})\setminus A_0, A_2 = W(a), B_1 = W(b^{-1}), B_2 = W(b)$.
\end{proof}

\begin{prop} If $G$ is $G$-paradoxical and acts on $X$ without fixed points, then $X$ is $G$-paradoxical.
\end{prop}

\begin{lem} $SO(3)$ contains a free group of rank $2$.
\end{lem}
\begin{proof} Let $\sigma, \tau$ be the matrices
\[
\sigma = \frac{1}{3}\begin{pmatrix} 1 & 2\sqrt{2} & 0\\ -2\sqrt{2} & 1 & 0\\ 0 & 0 & 3 \end{pmatrix}, \;\;\; \tau = \frac{1}{3}\begin{pmatrix} 3 & 0 & 0\\ 0 & 1 & -2\sqrt{2}\\ 0 & 2\sqrt{2} & 1 \end{pmatrix}.
\]
It's easy to check by induction on the length of $w$ that if $w$ is a word of length $k$ ending with $\sigma$, then $w\cdot (1\ 0\ 0)^T = \frac{1}{3^k} (a\ b\sqrt{2}\ c)^T$ with $3 \nmid b$, and that if $w$ starts with $\sigma^{\pm}$ then $a \equiv \pm b \pmod{3}$ and $c \equiv 0 \pmod{3}$, while if $w$ begins with $\tau^{\pm}$ then $c \equiv \pm b \pmod{3}$ and $a \equiv 0 \pmod{3}$. Thus $\sigma, \tau$ generate a free group of rank $2$.
\end{proof}

\begin{prop} If $E$ is a subset of $S^2$ with $|E| < 2^{\aleph_0}$, then $S^2$ is equidecomposable with $S^2\setminus E$.
\end{prop}
\begin{proof} We just need to find a rotation $\rho$ with $\rho^n(E) \cap E = \emptyset$ for all $n > 0$. We take the axis to be any line through the origin which doesn't pass through any point of $E$, and then choose the angle of the rotation avoiding $|E|\times|E|\times|\NN|$ bad angles.
\end{proof}

\begin{cor}[Banach-Tarski Paradox] Any ball in $\RR^3$ is paradoxical.
\end{cor}

\begin{cor}[Strong Banach-Tarski Paradox] If $A, B \subset \RR^3$ have nonempty interior and are bounded, then they are equidecomposable.
\end{cor}

% TODO: Amenable groups?

\begin{defn} A set of subsets $\gS$ of $X$ is a $\sigma$-\emph{algebra} over $X$ if $\gS$ satisfies: $\emptyset \in \gS$, $\forall A \in \gS$ we have $X\setminus A \in \gS$, and for any sequence $(A_n)_{n \in \NN}$ of elements of $\gS$ we have $\cup_n A_n \in \gS$.
\end{defn}

\begin{prop} If $\gS$ is a $\sigma$-algebra and $\gS$ is infinite, then $|\gS| \ge 2^{\aleph_0}$. If $\gS$ is generated by at most $2^{\aleph_0}$ sets, then $|\gS| \le 2^{\aleph_0}$ (more generally, if $\gS$ is generated by $\kappa$ sets, then $|\gS| \le \kappa^{\aleph_0}$).
\end{prop}

\begin{defn} If $X$ is a topological space, the \emph{Borel} $\sigma$-algebra is the smallest $\sigma$-algebra containing the open subsets of $X$ (some authors replace ``open'' by ``compact'' in this definition).
\end{defn}

\begin{prop} If $X$ is metric, then the Borel $\sigma$-algebra can be generated from the open sets by iteratively taking closure under countable unions and intersections at most $\omega_1$ times.
\end{prop}
\begin{proof} Every open subset of $X$ is a countable union of closed subsets of $X$, and $\omega_1$ has uncountable cofinality.
\end{proof}

\begin{cor} The Borel $\sigma$-algebra on $\RR$ has cardinality $2^{\aleph_0}$.
\end{cor}

\begin{prop} If $E$ is in the $\sigma$-algebra generated by $\cA \subseteq \cP(X)$, then there is a countable subset $\{A_1, A_2, ...\}$ of $\cA$ such that $E$ is in the $\sigma$-algebra generated by $A_1, A_2, ...$. In particular, $E$ can be written as a disjoint union of at most $2^{\aleph_0}$ countable intersections of elements of $\cA$.
\end{prop}

\begin{cor}[Nedoma's Pathology]\label{nedoma} If $|X| > 2^{\aleph_0}$, then the set $\Delta_X = \{(x,x) \mid x \in X\}$ is not in the $\sigma$-algebra generated by the collection of all rectangles $E\times F$, where $E,F$ are arbitrary subsets of $X$.
\end{cor}

\begin{defn} $\mu:\gS \rightarrow [0,\infty]$ is a \emph{measure} if $\mu(\emptyset) = 0$ and $\mu(\cup_{i=1}^\infty E_i) = \sum_{i=1}^\infty \mu(E_i)$ whenever $E_i \in \gS$ and $E_i \cap E_j = \emptyset$ for all $i \ne j$. $(X,\gS, \mu)$ is called a \emph{measure space} if $\gS$ is a $\sigma$-algebra over $X$ and $\mu : \gS \rightarrow [0,\infty]$ is a measure.
\end{defn}

\begin{prop} There is no translation invariant measure $\mu$ on the collection of all subsets of $\RR$ which satisfies $\mu([0,1]) = 1$.
\end{prop}
\begin{proof} Let $G$ be any additive subgroup of $\RR$ which contains $\ZZ$ and has $[G:\ZZ] = \aleph_0$ (we could take $G = \QQ$, $G = \ZZ[\sqrt{2}]$, etc.). Let $A$ be a set of representatives of $\RR/G$ which are all in $[0,1]$. Then there is a set $X \subset G$ with $|X| = \aleph_0$ such that $[0,1] \subseteq A+X \subseteq [-1,2]$. Thus $\mu(A) \le \frac{\mu([-1,2])}{n} = \frac{3}{n}$ for all $n > 0$, so $\mu(A) = 0$, so $\mu([0,1]) = 0$ by countable additivity.
\end{proof}

\begin{prop} If $\mu$ is a measure and $E_1 \subseteq E_2 \subseteq \cdots$ are measurable, then $\mu(\cup_{i=1}^\infty E_i) = \sup_i \mu(E_i)$. If $F_1 \supseteq F_2 \supseteq \cdots$ are measurable and $\mu(F_1) < \infty$, then $\mu(\cap_{i=1}^\infty F_i) = \inf_i \mu(F_i)$.
\end{prop}

\begin{defn} A set $E$ is $\sigma$-\emph{finite} with respect to a measure $\mu$ if $E$ can be written as a countable union of sets with finite $\mu$-measure. We say that $\mu$ is $\sigma$-finite if the whole space $X$ is $\sigma$-finite with respect to $\mu$. We say that $\mu$ is \emph{decomposable} if $X$ can be written as a disjoint union of $\sigma$-finite subsets $X_i$ such that for any $A \subseteq X$, $A$ is measurable iff $A \cap X_i$ is measurable for all $i$, and $\mu(A) = \sum_{i \in I} \mu(A\cap X_i)$.
\end{defn}

\begin{defn} A \emph{signed measure} is a map $\mu:\gS \rightarrow [-\infty,\infty]$ which is countably additive (and doesn't take both $\infty, -\infty$ as values).
\end{defn}

\begin{thm}[Hahn decomposition Theorem]\label{hahn-decomposition} If $\mu$ is a signed measure, then there exist measurable sets $P,N$ such that $P\cup N = X, P \cap N = \emptyset$, and for all $E \subseteq P$ measurable we have $\mu(E) \ge 0$, while for all $E \subseteq N$ measurable we have $\mu(E) \le 0$. This decomposition is unique up to null sets.
\end{thm}
\begin{proof} Assume WLOG that $\mu$ doesn't take the value $-\infty$. Say a measurable set is \emph{negative} if every measurable subset has measure $\le 0$. First we show that for any measurable $D$ with $\mu(D) \le 0$ there is a negative set $A \subseteq D$ with $\mu(A) \le \mu(D)$: define a sequence of sets $A_n$, $A_0 = D$, each $A_{n+1}$ given by removing a set of positive measure from $A_n$ whose measure is at least half as large as the $\sup$ of measures of subsets (if finite), or at least $1$ otherwise, and take $A = \cap_n A_n$. Next, we define N by making a sequence $N_n$ with $N_0 = \emptyset$, and $N_{n+1}$ given by adding a negative set to $N_n$ whose measure is at least half as negative as the $\inf$ of measure of subsets (if finite), or at most $-1$ otherwise, and take $N = \cup_n N_n$.
\end{proof}

\begin{thm}[Jordan decomposition Theorem] If $\mu$ is a signed measure, there is a unique decomposition $\mu = \mu^+ - \mu^-$ where $\mu^+, \mu^-$ are positive measures (at least one of which is finite), such that $\mu^+(E)$ is $0$ for any negative set $E$ and $\mu^-$ is $0$ for any positive set $E$.
\end{thm}

\begin{defn} If $\mu$ is a signed measure and $\mu = \mu^+ - \mu^-$ is its Jordan decomposition, then we set $|\mu| = \mu^+ + \mu^-$.
\end{defn}

\begin{defn} A \emph{complex measure} is a countably additive function $\mu:\gS \rightarrow \CC$. Equivalently, it is a complex combination of finite measures.
\end{defn}

\begin{defn} If $\mu$ is a complex measure, we define the \emph{total variation} of $\mu$ to be the positive measure $|\mu|$ given by $|\mu|(E) = \sup\{\sum_{i=1}^n |\mu(E_i)| \mid E = E_1 \dotcup \cdots \dotcup E_n\}$.
\end{defn}

\begin{defn} If $\mu, \nu$ are (possibly signed) measures, then $\mu$ is \emph{absolutely continuous} with respect to $\nu$, written $\mu \ll \nu$, if $|\nu|(A) = 0 \implies |\mu|(A) = 0$.
\end{defn}

\begin{prop}\label{absolutely-continuous} If $\mu$ is finite, then $\mu \ll \nu$ iff for all $\epsilon > 0$ there exists a $\delta > 0$ such that $|\nu|(A) < \delta \implies |\mu|(A) < \epsilon$.
\end{prop}
\begin{proof} Assume $\mu, \nu$ are positive. For every $n \ge 1$, let $n\nu - \mu$ have Hahn decomposition $(P_n,N_n)$, and let $N = \bigcap_n N_n$. Since $n\nu(N) \le \mu(N)$ for all $n$ and $\mu(N) < \infty$, we have $\nu(N) = 0$, so we must have $\mu(N) = 0$ by $\mu \ll \nu$. Thus there is some $n$ such that $\mu(N_n) < \frac{\epsilon}{2}$, and we can take $\delta = \frac{\epsilon}{2n}$: if $\nu(A) < \delta$, then $\mu(A) = \mu(A\cap P_n) + \mu(A\cap N_n) \le n\nu(A) + \mu(N_n) < \frac{\epsilon}{2} + \frac{\epsilon}{2}$.
\end{proof}

\begin{defn} We say that two (possibly signed or complex) measures $\mu,\nu$ on $X$ are \emph{singular}, written $\mu \perp \nu$, if there are measurable sets $A,B$ with $A \cup B = X$ such that $B$ is $\mu$-null and $A$ is $\nu$-null.
\end{defn}

\begin{thm}[Lebesgue decomposition Theorem]\label{lebesgue-decomposition} If $\mu, \nu$ are (possibly signed) $\sigma$-finite measures over $X$, then there is a unique pair of $\sigma$-finite measure $\mu_{ac}, \mu_s$ such that $\mu = \mu_{ac} + \mu_s$, $\mu_{ac} \ll \nu$, and $\mu_s \perp \nu$.
\end{thm}
\begin{proof} We just need to prove this in the finite, unsigned case. Let $\cN$ be the collection of $\nu$-null sets. Define $\mu_{ac}$ by
\[
\mu_{ac}(A) = \inf_{N \in \cN} \mu(A\setminus N).
\]
$\mu_{ac}$ is clearly nonnegative and countably additive, and we clearly have $\mu_{ac} \ll \nu$. Set $\mu_s = \mu - \mu_{ac}$, taking $A = X$ and noting that the infimum must actually be attained, we see that there is a $\nu$-null set $N$ such that $\mu_s(X\setminus N) = 0$, so $\mu_s \perp \nu$.

For uniqueness, suppose that $\mu = \mu_1 + \mu_2$ with $\mu_1 \ll \nu, \mu_2 \perp \nu$. Since $\mu_1 \le \mu$ and $\mu_1 \ll \nu$, we have
\[
\mu_1(A) = \inf_{N \in \cN} \mu_1(A\setminus N) \le \inf_{N \in \cN} \mu(A\setminus N) = \mu_{ac}(A),
\]
so $\mu_1 \le \mu_{ac}$. Thus $\mu_{ac} - \mu_1 = \mu_2 - \mu_s$ is both $\nu$-absolutely continuous and $\nu$-singular, so $\mu_1 = \mu_{ac}$.
\end{proof}

\subsubsection{Constructing measures}

\begin{defn} On any set, the \emph{counting measure} takes every finite set to its size and every infinite set to $\infty$. If $S = \{s_1, ...\}$ is a countable subset of $X$ and $a_1, ... \in [0,\infty]$, then the \emph{discrete measure} $\sum_i a_i\delta_{s_i}$ is given by $E \mapsto \sum_{s_i \in E} a_i$. More generally, if $f : X \rightarrow [0,\infty]$, we can define a measure $A \mapsto \sum_{a \in A} f(a)$, where the sum over $A$ is defined to be the supremum of all the sums over finite subsets of $A$.
\end{defn}

\begin{defn} A measure space $(X, \gS, \mu)$ is \emph{complete} if every subset of a null set (that is, a set with measure $0$) is in $\gS$. If $Z$ is the collection of all subsets of null sets, then define $\Sigma_0$ to be the $\sigma$-algebra generated by $\Sigma$ and $Z$, and $\mu_0(C) = \inf\{\mu(D) \mid C \subseteq D \in \Sigma\}$, and define the \emph{completion} of $(X,\gS,\mu)$ to be $(X,\gS_0,\mu_0)$.
\end{defn}

\begin{prop} The completion of a measure space is always a complete measure space, and in fact $\Sigma_0 = \{A \cup B \mid A \in \Sigma, B \in Z\}$.
\end{prop}

\begin{defn} $\varphi : \cP(X) \rightarrow [0,\infty]$ is an \emph{outer measure} if $\varphi(\emptyset) = 0$, $A \subseteq B \implies \varphi(A) \le \varphi(B)$, and for any sequence $(A_n)_{n\in \NN}$ we have have $\varphi(\cup_{i=1}^\infty A_i) \le \sum_{i=1}^\infty \varphi(A_i)$.
\end{defn}

\begin{defn} If $\varphi$ is an outer measure over $X$, we say that $E$ is $\varphi$-\emph{measurable} if $\forall A \subseteq X$, we have $\varphi(A) = \varphi(A\cap E) + \varphi(A\cap E^c)$. We write $\gS_\varphi$ for the collection of all $\varphi$-measurable sets.
\end{defn}

\begin{thm} If $\varphi$ is an outer measure, then $\gS_\varphi$ is a $\sigma$-algebra, and the restriction of $\varphi$ to $\gS_\varphi$ is a complete measure.
\end{thm}
\begin{proof} If $E_i \in \gS_\varphi$ are pairwise disjoint and $E = \cup_{i=1}^\infty E_i$, then for any $A$ we have
\[
\varphi(A) \le \varphi(A\cap E^c) + \varphi(A\cap E) \le \varphi(A\cap E^c) + \sum_{i=1}^\infty \varphi(A\cap E_i) = \sup_n \Big(\varphi(A\cap E^c) + \sum_{i=1}^n \varphi(A\cap E_i)\Big) \le \varphi(A).
\]
Taking $A = E$ shows that $\varphi(E) = \sum_{i=1}^\infty \varphi(E_i)$.
\end{proof}

\begin{defn} An outer measure $\varphi$ is \emph{regular} if for every set $E$ there exists a $\varphi$-measurable set $A \supseteq E$ with $\varphi(E) = \varphi(A)$. (Note: don't confuse regular outer measures with ``outer regular'' Borel measures in the topological setting!)
\end{defn}

\begin{prop} If $\varphi$ is an outer measure and $\varphi(A) = 0$, then $A$ is $\varphi$-measurable. More generally, if $B \subseteq A$ with $B$ $\varphi$-measurable, $\varphi(A) < \infty$, and $\varphi(A) = \varphi(B)$, then $A$ is $\varphi$-measurable.
\end{prop}

\begin{prop} If $\varphi$ is a regular outer measure, $A \subseteq B$ with $B$ $\varphi$-measurable, $\varphi(B) < \infty$, $\varphi(A)$, and $\varphi(B) = \varphi(A) + \varphi(B\setminus A)$, then $A$ is $\varphi$-measurable.
\end{prop}

\begin{defn} If $X$ is a metric space and $\varphi$ is an outer measure over $X$, we say that $\varphi$ is a \emph{metric outer measure} if $d(E,F) > 0 \implies \varphi(E \cup F) = \varphi(E) + \varphi(F)$.
\end{defn}

\begin{thm} If $\varphi$ is a metric outer measure, then all Borel sets are $\varphi$-measurable.
\end{thm}
\begin{proof} If $U$ is open, let $U_n = \{x \in U \mid B(x,\frac{1}{n}) \subseteq U\}$, and note that for any $n$, $d(U_n, U_{n+1}^c) \ge \frac{1}{n(n+1)} > 0$. For any $A$ with $\varphi(A) < \infty$ we then have
\[
\sum_{n\text{ odd}} \varphi(A \cap (U_{n+1}\setminus U_n)) \le \varphi(A) < \infty,
\]
and similarly for $n$ even, so the tails of the sum go to zero. Then for any $A$ we have
\[
\varphi(A) \le \varphi(A \cap U^c) + \varphi(A \cap U) \le \inf_n \Big(\varphi(A\cap U^c) + \varphi(A\cap U_n) + \sum_{m \ge n} \varphi(A\cap (U_{m+1}\setminus U_m))\Big) \le \varphi(A).\qedhere
\]
\end{proof}

\begin{defn} A $G_\delta$ set is any countable intersection of open sets, and an $F_\sigma$ set is any countable union of closed sets.
\end{defn}

\begin{prop} In a metric space, every closed set is a $G_\delta$ set and every open set is an $F_\sigma$ set.
\end{prop}

\begin{prop} If $X$ is a topological space, $Y$ a metric space, and $f:X\rightarrow Y$ is any function, then the set of points of continuity of $f$ is a $G_\delta$ set.
\end{prop}
\begin{proof} Let $C$ be the set of points of continuity of $f$, and for each $c \in C$ and each $n \in \mathbb{N}^+$, pick an open set $c \in U_n^c \subseteq X$ such that $x \in U_n^c \implies d_Y(f(x),f(c)) < \frac{1}{n}$. Then $C = \cap_n \cup_{c\in C} U_n^c$.
\end{proof}

\begin{defn} A collection of sets $S$ is a \emph{semi-ring} if $\emptyset \in S$, for any $A,B \in S$ we have $A \cap B \in S$, and for any $A, B \in S$ there exists $n$ and pairwise disjoint $C_1, ..., C_n \in S$ such that $A \setminus B = \cup_{i=1}^n C_i$.
\end{defn}

\begin{defn} If $S$ is a collection of sets, then a map $\mu : S \rightarrow [0,\infty]$ is a \emph{pre-measure} if $\mu(\emptyset) = 0$ and for any sequence $A_n$ of pairwise disjoint sets in $S$ such that $\cup_{i=1}^\infty A_i \in S$, we have $\mu(\cup_{i=1}^\infty A_i) = \sum_{i=1}^\infty \mu(A_i)$.
\end{defn}

\begin{thm}[Carath\'eodory Extension Theorem]\label{caratheodory-extension} If $S$ is a semi-ring of subsets of $X$ and $\mu_0 : S \rightarrow [0,\infty]$ is a pre-measure, then if we define $\mu^*$ by
\[
\mu^*(E) = \inf \Big\{\sum_{i=1}^\infty \mu_0(A_i) \mid A_i \in S,\ E \subseteq \bigcup_{i=1}^\infty A_i\Big\},
\]
then $\mu^*$ is an outer measure over $X$ with $\mu^*(A) = \mu_0(A)$ for all $A\in S$, and $S \subseteq \gS_{\mu^*}$.
\end{thm}

\begin{defn} A pre-measure $\mu : S \rightarrow [0,\infty]$ with $S$ a collection of subsets of $X$ is $\sigma$-\emph{finite} if there exists a sequence $A_n \in S$ with $\mu(A_i) < \infty$ and $X = \cup_{i=1}^\infty A_i$.
\end{defn}

\begin{thm}[Hahn-Kolmogorov] If $\mu_0$ is a pre-measure on a semi-ring $S$, then it extends to a measure $\mu$ on the $\sigma$-algebra $\gS$ generated by $S$. If $\mu_0$ is $\sigma$-finite, then this extension is unique.
\end{thm}
\begin{proof} Let $\mu^*$ be the associated outer measure from the Carath\'eodory extension theorem, and suppose $\mu'$ is a different measure extending $\mu$ on $\gS' \supseteq S$. Then for any $E \in \gS' \cap \gS_{\mu^*}$, we clearly have $\mu'(E) \le \mu^*(E)$. By $\sigma$-finiteness and the fact that $\mu'$ is countably additive, we can assume WLOG that $\mu^*(X) = \mu'(X) < \infty$, but then $\mu'(E^c) \le \mu^*(E^c)$ implies $\mu'(E) = \mu^*(E)$ since $E$ is $\mu^*$-measurable.
\end{proof}

\begin{prop} Let $\mu_0, \mu^*, \mu, S, \gS, \gS_{\mu^*}$ be as above. If $\mu_0$ is $\sigma$-finite, then $\gS_{\mu^*}$ is the completion of $\gS$ - in fact, for any $E \in \gS_{\mu^*}$, there is a countable intersection of countable unions of elements of $S$ which contains $E$ and differs from it in a null set.
\end{prop}

\begin{thm}[Lebesgue outer measure] Let $S$ be the collection of half-open intervals $[a,b)$ for $a \le b \in \RR$, and define $\gl_0 : S \rightarrow [0,\infty)$ by $\gl_0([a,b)) = b-a$. Then $S$ is a semi-ring, $\gl_0$ is a pre-measure, and the associated outer measure $\gl^*$ is a translation-invariant metric outer measure over $\RR$ with $\gl^*([0,1]) = 1$.
\end{thm}
\begin{proof} Suppose that $[a,b) = \cup_{i=1}^\infty A_i$, where the $A_i$ are pairwise disjoint half-open intervals. Then the set of left endpoints of the $A_i$ is well-ordered (any descending sequence must have a limit in $[a,b)$, and this limit must be contained in some $A_i$), so we can show by well-founded induction that if $A_i = [c,d)$, then $\sum_{A_j < A_i} \gl_0(A_j) = c-a$.

Alternate proof: Let $A' = [a,b-\epsilon]$, and if $A_i = [c_i,d_i)$ let $A_i' = (c_i - \epsilon/2^i, d_i)$. Then by compactness, some finite subset of the $A_i'$s cover $A'$.
\end{proof}

\begin{defn} If $\gl^*$ is constructed as above, then a set is called \emph{Lebesgue-measurable} if it is in $\gS_{\gl^*}$, and $\gl^*\mid_{\gS_{\gl^*}}$ is called the \emph{Lebesgue measure}, and written as $\gl$.
\end{defn}

\begin{thm}[Lebesgue-Stieltjes measure]\label{lebesgue-stieltjes} If $I$ is an interval and $g: I \rightarrow \RR$ is monotone increasing, set $g_-(x) = \sup_{y < x} g(y)$, then there is a unique Borel measure $\mu_g$ such that $\mu_g([a,b)) = g_-(b) - g_-(a)$. If $g$ is continuous, then for any $E$ we have $\mu_g(E) = \lambda(g(E))$.
\end{thm}

\begin{defn} If $g$ has bounded variation, then we define the \emph{signed Lebesgue-Stieltjes measure} $\mu_g$ by writing $g = g_1 - g_2$ with $g_1, g_2$ monotone increasing, and $\mu_g = \mu_{g_1} - \mu_{g_2}$.
\end{defn}

\begin{defn} A Borel measure $\mu$ is \emph{locally finite} if every point has an open neighborhood of finite measure. It is \emph{inner regular} on $B$ if $\mu(B)$ is the supremum of $\mu(K)$ over all compact $K \subseteq B$. It is \emph{outer regular} if for all Borel sets $B$, $\mu(B)$ is the infimum of $\mu(U)$ over all open $U$ containing $B$. A measure is \emph{Radon} if it is inner regular on open sets, outer regular, and locally finite.
\end{defn}

\begin{prop} Every locally finite Borel measure over $\RR$ is a Lebesgue-Stieltjes measure, and every Lebesgue-Stieltjes measure is a Radon measure. More generally, every locally finite Borel measure on $\RR^n$ is Radon.
\end{prop}

\begin{thm}[Besicovitch Covering Theorem for Radon Measures]\label{besicovitch-radon} If $E$ is a bounded subset of $\mathbb{R}^n$ and $\mu$ is a Radon measure on $\mathbb{R}^n$ with associated outer measure $\mu^*$, and if $\mathcal{B}$ is a collection of closed balls such that every point in $E$ is the center of an arbitrarily small ball of $\mathcal{B}$, then there exists a countable collection of disjoint balls $\{B_i\} \subseteq \mathcal{B}$ such that $\mu^*(E\setminus \cup_i B_i) = 0$.
\end{thm}
\begin{proof} By the Besicovitch Covering Lemma \ref{besicovitch}, we can find a finite number $c_n$ of families $\mathcal{B}_i \subseteq \mathcal{B}$ of disjoint balls such that $E \subseteq \cup_i \cup_{B \in \mathcal{B}_i} B$. Then for some $i$ we must have $\mu^*(E\cap \cup_{B\in \mathcal{B}_i}B) \ge \mu^*(E)/c_n$. Pick some finite subset $B_1, ..., B_k$ of $\mathcal{B}_i$ such that $\mu^*(E\cap \cup_{j\le k} B_j) \ge \mu^*(E)/2c_n$. Now replace $E$ by $E\setminus \cup_{j \le k} B_j$ and replace $\mathcal{B}$ by the set of balls of $\mathcal{B}$ which do not intersect the closed set $\cup_{j \le k} B_j$, and iterate.
\end{proof}

\begin{thm}[Product measures]\label{product-measure} If $\mu, \nu$ are pre-measures on semi-rings $S,T$, respectively, then the collection of rectangles $S\times T$ is a semi-ring, and $\mu\times \nu$ is a pre-measure on $S\times T$.
\end{thm}
\begin{proof} Suppose $E\times F \in S\times T$ is a countable union of disjoint rectangles $E_i \times F_i$. We'll show that for any $M < \mu(E)$ and $N < \nu(F)$, we have $MN \le \sum_i \mu(E_i)\nu(F_i)$. Let $A_n = \{x \in E \mid \sum_{i=1}^n 1_{x \in E_i}\cdot\nu(F_i) \ge N\}$. Each $A_n$ is a finite union of elements of $S$, and $\cup_n A_n = E$ since for each $x \in E$, the collection of $F_i$s with $x \in E_i$ is disjoint and covers $F$, so some finite subset of them must have measure at least $N$. Thus there is some $n$ such that $\mu(A_n) \ge M$, and for this $n$ we have $MN \le \sum_{i=1}^n \mu(E_i)\nu(F_i)$.
\end{proof}

\begin{thm}[Infinite products] Let $I$ be any index set. If $\mu_i$ are pre-measures on semi-rings $S_i$, such that each $S_i$ has an element $X_i$ with $\mu_i(X_i) = 1$, and if we let $S = \prod_{i\in I}' S_i$ be the set of rectangles $\prod_{i \in I} A_i$ such that $A_i = X_i$ for all but finitely many $i$ and define $\mu = \prod_i \mu_i$, then $S$ is a semi-ring and $\mu$ is a pre-measure on $S$.
\end{thm}
\begin{proof} Suppose that $A = \cup_{n=1}^\infty A_n$ with $A, A_n \in S$ and the $A_n$s disjoint, but that $\mu(A) > \sum_n \mu(A_n)$. Each $A_n$ only has finitely many coordinates $i$ which are not equal to $X_i$, so at most countably many coordinates in $I$ are relevant - rename these relevant coordinates as $1, 2, ...$. Write $A = E \times F$, $A_n = E_n \times F_n$, with $E, E_n \in S_1$ and $F, F_n \in \prod_{i \ne 1}' S_i$, and write $\mu^1 = \prod_{i \ne 1} \mu_i$. By the argument for the finite case, there is some $x_1 \in E$ such that $\mu^1(F) > \sum_n 1_{x_1 \in E_n}\cdot \mu^1(F_n)$. Continuing inductively, we find a sequence of coordinates $x_1, x_2, ...$ such that for each $k$, when we restrict the first $k$ coordinates to be $x_1, ..., x_k$, the two sides don't add up. But then no point with $(x_1, x_2, ...)$ as the relevant countably many coordinates can be an element of any $A_n$ (take $k$ to be larger than the finitely many coordinates $i$ of $A, A_n$ which are not equal to $X_i$), contradicting the assumption $A = \cup_n A_n$.
\end{proof}

\begin{cor}[Lebesgue measure on $\RR^n$] For every $n$, there is a translation-invariant metric outer measure $\gl^*$ on $\RR^n$ with $\gl^*([0,1]^n) = 1$. If $T$ is a linear transformation and $A \subseteq \RR^n$, then $\gl^*(T(A)) = |\det(T)|\gl^*(A)$. The associated measure $\gl$ is a Radon measure.
\end{cor}
\begin{proof} For the statement about linear transformations, it's enough to check this for shear and stretch transformations in the case $A$ is a box, and this can done done using a standard dissection argument (the pieces are Borel sets).
\end{proof}

\begin{defn} If $X,Y$ are measure spaces with measures $\mu, \nu$, then $X\times Y$ has a measure $\mu\times \nu$ given by applying the Carath\'eodory extension Theorem \ref{caratheodory-extension} to the product pre-measure contructed in Theorem \ref{product-measure} - this measure is called the \emph{maximal product measure} on $X\times Y$.
\end{defn}

\begin{prop}\label{product-null} If $A \subseteq X\times Y$ is $\mu\times\nu$-null, then the set of $y \in Y$ such that $A_y = \{x \in X \mid (x,y) \in A\}$ is not $\mu$-null is $\nu$-null.
\end{prop}
\begin{proof} Pick $\epsilon > 0$, and let $E$ be the set of $y \in Y$ such that $\mu(A_y) > \epsilon$. If $A \subseteq \cup_{n=1}^\infty R_n$ such that the $R_n$ are measurable rectangles, and $E_k$ is the set of $y$ such that $\mu((\cup_{n=1}^k R_k)_y) > \epsilon$, then $\cup_k E_k = E$, so if $\nu(E) > \delta$ then some $\nu(E_k) > \delta/2$, so $\mu\times\nu(\cup_n R_n) > \epsilon\delta/2$.
\end{proof}

\begin{thm}[Cavalieri Principle] If $X,Y$ are $\sigma$-finite measure spaces and $A,B \subseteq X\times Y$ are measurable with $\mu(A_y) = \mu(B_y)$ for $\nu$-almost every $y \in Y$, then $\mu\times\nu(A) = \mu\times\nu(B)$.
\end{thm}
%\begin{proof} TODO: find a proof that doesn't use integrals.
%\end{proof}

\begin{ex} To see $\sigma$-finiteness is necessary, take $X$ to be $[0,1]$ with counting measure, $Y$ to be $[0,1]$ with Lebesgue measure, $A$ to be $\{0\}\times Y$, and $B$ to be the diagonal.
\end{ex}

\begin{thm}[Lebesgue Density Theorem]\label{lebesgue-density} If $E \subseteq \RR^n$, then for Lebesgue-a.e. $x$ in $E$ we have
\[
\lim_{r \rightarrow 0} \frac{\gl^*(E\cap B_r(x))}{\gl(B_r(x))} = 1.
\]
\end{thm}
\begin{proof} Let $A_t$ be the set of points such that the left hand side (with a $\liminf$ instead) is less than $1-t$, and let $U_\epsilon$ be an open set containing $A_t$ with $\gl^*(U_\epsilon \setminus A_t) \le \epsilon$. Then for each point $x$ in $A_t$, we can find an $r$ such that the left hand side of the above is at most $1-t$ and such that $B_r(x) \subseteq U_\epsilon$. Now apply the Vitali Covering Lemma to get a collection $(B_i)_{i \in I}$ of disjoint balls contained in $U_\epsilon$ such that $A_t \subseteq \cup_i 5B_i$. Then since $\cup_i B_i \subseteq U_\epsilon$, we have
\[
\gl(\cup_i B_i)-\epsilon \le \gl^*(A \cap (\cup_i B_i)) \le \gl^*(E\cap (\cup_i B_i)) \le \sum_i (1-t)\gl(B_i) = (1-t)\gl(\cup_i B_i),
\]
so $\gl(\cup_i B_i) \le \epsilon/t$, and since $A_t \subseteq \cup_i 5B_i$ we get $\gl^*(A_t) \le 5^n\epsilon/t$. Since $\epsilon > 0$ was arbitrary, $\gl^*(A_t) = 0$.
\end{proof}

\begin{defn} If $X$ is a metric space and $S \subseteq X$, we set
\[
H^d_\delta(S) = \inf\Big\{\sum_{i=1}^\infty \diam(U_i)^d \mid S \subseteq \bigcup_{i=1}^\infty U_i,\ \diam(U_i) < \delta\Big\}
\]
and
\[
H^d(S) = \sup_{\delta > 0} H^d_\delta(S).
\]
This is a metric outer measure, called the \emph{Hausdorff measure}.
\end{defn}

\begin{thm} In $\RR^n$, we have $H^n(B) = 2^n$, where $B$ is the unit ball.
\end{thm}
\begin{proof} This follows from the isodiametric inequality: the volume of a set of diameter $2$ is at most the volume of the unit ball. Suppose that $K$ has diameter $2$, then $K-K \subseteq 2B$, so by Brunn-Minkowski we have $\gl(K) \le \gl(\frac{1}{2}(K-K)) \le \gl(B)$.
\end{proof}

\begin{defn} If $X$ is a metric space and $S \subseteq X$, we define the \emph{Hausdorff content} of $S$ to be
\[
C^d(S) = \inf\Big\{\sum_{i=1}^\infty r_i^d \mid S \subseteq \bigcup_{i=1}^\infty B(x_i,r_i)\Big\}.
\]
\end{defn}

\begin{prop} If $X$ is a metric space and $S \subseteq X$, then $C^d(S) = 0$ iff $H^d(S) = 0$.
\end{prop}

\begin{defn} If $X$ is a metric space and $S \subseteq X$, the \emph{Hausdorff dimension} of $S$ is defined to be the infimum of the set of $d$ such that $C^d(S) = 0$.
\end{defn}

\begin{prop} If $X$ is a metric space and $S \subseteq X$, then there is a $G_\delta$ set which contains $S$ and has the same Hausdorff dimension as $S$.
\end{prop}

\begin{prop} If $X$ is a metric space and $S \subseteq X$, then there is a $G_\delta$ set which contains $S$ and has the same Hausdorff measure $H^d$ as $S$ (so $H^d$ is a regular outer measure). If additionally $S$ is $H^d$-measurable and $H^d(S) < \infty$, then there is an $F_\sigma$ set contained in $S$ with the same Hausdorff measure as $S$.
\end{prop}
\begin{proof} For the first part, note that in the definition of $H^d_\delta(S)$ we may restrict the covers to be covers by open sets without changing the $\inf$, and take an intersection of open covers over a sequence of $\delta$s going to $0$. For the second part, let $\cap_i O_i \supseteq S$ be the $G_\delta$ set from the first part, and write each $O_i$ as an $F_\sigma$ set by $O_i = \cup_j C_{ij}$. For each $i$, find $j_i$ such that $H^d(S\setminus C_{ij_i}) < \epsilon/2^i$, and let $C_\epsilon = \cap_i C_{ij_i}$. Then $H^d(C_\epsilon) > H^d(S) - \epsilon$, and $H^d(C_\epsilon\setminus S) \le H^d(\cap_i O_i \setminus S) = 0$, so we can find an $H^d$-null $G_\delta$ set containing $C_\epsilon\setminus S$, and removing it from $C_\epsilon$ we get an $F_\sigma$ set $C_\epsilon' \subseteq S$. Now take a union over a sequence of $\epsilon$s going to $0$.
\end{proof}

\begin{thm}[Vitali Covering Theorem for Hausdorff Measure]\label{vitali-hausdorff} If $E$ is a subset of a metric space, and $\mathcal{V}$ is a collection of sets such that every point of $E$ is contained in an element of $\mathcal{V}$ of arbitrarily small nonzero diameter, then there is a countable disjoint subcollection $\{U_i\} \subseteq \mathcal{V}$ such that either $H^d(E\setminus \cup_i U_i) = 0$ or $\sum_i \diam(U_i)^d = \infty$.

Furthermore, if $H^d(E) < \infty$, then for any $\epsilon > 0$ we may choose this subcollection such that $H^d(E) \le \sum_i \diam(U_i)^d + \epsilon$.
\end{thm}
\begin{proof} Let $\rho$ be small, and assume WLOG that all diameters of sets in $\mathcal{V}$ are at most $\rho$. At each step, choose $U_i$ to be disjoint from $U_1, ..., U_{i-1}$ with diameter at least $\frac{1}{2}$ the $\sup$ of the diameters of such disjoint sets. For each $i$, let $B_i$ be a ball with center in $U_i$ and radius $3\diam(U_i)$, then $E\setminus \cup_{i \le k} U_i \subseteq \cup_{i > k} B_i$. If $\sum_i \diam(U_i)^d < \infty$, then $\diam(B_i) \rightarrow 0$ and the tails of $\sum_i \diam(B_i)^d$ go to $0$, so each $H^d_\delta(E\setminus \cup_i U_i) = 0$, etc.
\end{proof}

\begin{cor} An arbitrary union of full-dimensional closed convex subsets of $\mathbb{R}^n$ is Lebesgue measurable.
\end{cor}
\begin{proof} Reduce to the case where everything is contained in a big ball and all the convex sets $C$ satisfy $\diam(C)^n \le k\lambda(C)$ for some integer $k$, then apply the Vitali Covering Theorem \ref{vitali-hausdorff} to the collection of homothetic images of these convex sets which are contained within the union, to see that the union can be decomposed into a countable disjoint union of closed convex sets together with a set of measure $0$.
\end{proof}

For exceedingly large (in terms of cardinality) spaces, issues like Nedoma's pathology \ref{nedoma} require us to use a different approach to constructing measures, by making use of topological structure.

\begin{defn} If $X$ is a locally compact Hausdorff space, then a Borel measure $\mu$ is called a \emph{Radon measure} if it is locally finite, outer regular, and inner regular on open sets.
\end{defn}

\begin{defn} A Borel measure $\mu$ is called an \emph{inner Radon measure} if it is locally finite and inner regular on all Borel sets.
\end{defn}

\begin{prop}\label{radon-sigma-finite} If $X$ is Hausdorff, $\mu$ is a Radon measure, and $E$ is $\sigma$-finite, then $\mu(E) = \sup \{\mu(K) \mid K \subseteq E,\ K\text{ \rm compact}\}$. 
\end{prop}
\begin{proof} It's enough to prove this when $\mu(E) < \infty$. Take an open set $U \supseteq E$ with $\mu(U) < \infty$, take a compact $K \subseteq U$ with $\mu(U\setminus K) < \epsilon$, and take an open $V$ with $K \setminus E \subseteq V$, $\mu(V\setminus(K\setminus E)) < \epsilon$, then $K\setminus V$ is compact, contained in $E$, and $\mu(K\setminus V) > \mu(E) - 2\epsilon$.
\end{proof}

%\begin{defn} A \emph{field} of sets is a collection of sets which is closed under finite intersections, unions and complements. A \emph{content} on a field of sets $\cA$ is a function $\lambda : \cA \rightarrow [0,\infty]$, such that $\lambda(A)$ is increasing in $A$, and such that for any $A_1, A_2 \in \cA$ disjoint, we have $\lambda(A_1 \cup A_2) = \lambda(A_1) + \lambda(A_2)$.
%\end{defn}

%\begin{defn} Call a collection $\cK$ of compact subsets of a locally compact Hausdorff space $X$ \emph{splittable} if for any $K \in \cK$ and $U_1,U_2$ open with $K \subseteq U_1 \cup U_2$ there are $K_1,K_2 \in \cK$ with $K_i \subseteq U_i$ and $K \subseteq K_1\cup K_2$, and if additionally $\cK$ is closed under finite unions and for every $x \in X$ there is a $K\in \cK$ which is a neighborhood of $x$.
%\end{defn}

%\begin{prop} Suppose that $\cK$ is a collection of compact subsets of a locally compact Hausdorff space which is closed under finite unions and includes a neighborhood of every point, and that $\cB$ is a base of open subsets such that for any $K \in \cK$ and $U \in \cB$, we have $K \setminus U \in \cK$. Then $\cK$ is splittable.
%\end{prop}
%\begin{proof} Suppose $K$ is compact and $U_1,U_2$ open. Let $L_1 = K\setminus U_2, L_2 = K\setminus U_1$, then $L_1,L_2$ are disjoint compact sets of a Hausdorff space, so there are disjoint open sets $V_1, V_2$ with $L_i \subseteq V_i$ and such that each $V_i$ is a finite union of elements of $\cB$. Now take $K_1 = K\setminus V_2, K_2 = K \setminus V_1$.
%\end{proof}

%\begin{prop} Let $X,Y$ be locally compact Hausdorff spaces, then the collection of all finite unions of products of compact subsets of $X$ and compact subsets of $Y$ is a splittable collection of compact subsets of $X\times Y$.
%\end{prop}

%\begin{prop} Let $X_i$ be compact Hausdorff spaces for all $i \in I$, then the collection of all finite unions of products $\prod_{i \in I} K_i$ such that all $K_i$ are compact subsets of $X_i$ and all but finitely many $i$ have $K_i = X_i$ is a splittable collection of compact subsets of $\prod_{i \in I} X_i$.
%\end{prop}

%\begin{prop} If $\cK$ is a splittable collection of compact sets, then for any $K \subseteq U$ with $K$ compact and $U$ open, there is an $L \in \cK$ with $K \subseteq \inter(L)$ and $L \subseteq U$.
%\end{prop}
%\begin{proof} For any $x \in K$, there is $C_x \in \cK$ with $x \in \inter(C_x)$. Since $C_x \subseteq U\cup K^c$, there are $L_x, M_x \in \cK$ with $L_x \subseteq U, M_x \subseteq K^c$ and $C_x \subseteq L_x \cup M_x$. Thus $x \in \inter(C_x)\setminus M_x \subseteq \inter(L_x)$, and by compactness we can take $L$ to be a finite union of the sets $L_x$.
%\end{proof}

\begin{prop} In a locally compact Hausdorff space, if $K \subseteq U$ with $K$ compact and $U$ open, then there is a compact $L$ with $K \subseteq \inter(L)$ and $L \subseteq U$.
\end{prop}

\begin{defn} Call a collection $\cK$ of compact subsets of a locally compact Hausdorff space $X$ \emph{splittable} if $\cK$ is a local base of neighborhoods of $X$ which is closed under finite unions.
\end{defn}

\begin{prop} Suppose that $\cK$ is a splittable collection of compact subsets. Then for any $K$ compact and $U_1,U_2$ open with $K \subseteq U_1 \cup U_2$ there are $K_1,K_2 \in \cK$ with $K_i \subseteq U_i$ and $K \subseteq K_1\cup K_2$.
\end{prop}

\begin{defn} A \emph{content} on a splittable collection of compact sets $\cK$ is a function $\lambda : \cK \rightarrow [0,\infty)$, such that $\lambda(K)$ is increasing in $K$, $\lambda(K_1 \cup K_2) \le \lambda(K_1) + \lambda(K_2)$, and such that for any $K_1, K_2 \in \cK$ disjoint, we have $\lambda(K_1 \cup K_2) = \lambda(K_1) + \lambda(K_2)$. A content $\lambda$ is \emph{regular} if for any $K \in \cK$, we have $\lambda(K) = \inf \{\lambda(L) \mid K \subseteq \inter(L)\}$.
\end{defn}

\begin{thm}\label{content-measure} For every content $\lambda$ on a splittable collection of compact subsets of a locally compact Hausdorff space $X$, there is a unique Radon measure $\mu$ on $X$ such that for all open sets $U$ we have $\mu(U) = \sup\{\lambda(K) \mid K \subseteq U\}$. If $\lambda$ is a regular content, then $\mu$ extends $\lambda$.
\end{thm}
\begin{proof} Define $\mu$ on open sets as in the theorem statement, and define $\mu^* : \cP(X) \rightarrow [0,\infty]$ by $\mu^*(A) = \inf \{\mu(U) \mid A \subseteq U\}$. $\mu$ is finite on the interior of any compact set, so $\mu^*$ is locally finite.

First we show that $\mu^*$ is an outer measure: If $A = \cup_{n=1}^\infty A_n$, then pick $U_n$ open with $A_n \subseteq U_n$ and $\mu^*(U_n) \le \mu^*(A_n) + \epsilon/2^n$, and let $U = \cup_n U_n$. Pick $K \subseteq U$ compact with $\mu(U) \le \lambda(K) + \epsilon$, then some finite subset of the $U_n$ cover $K$, say $U_1, ..., U_k$. We just need to show that $\lambda(K) \le \sum_{i=1}^k \mu(U_i)$, and this follows if we can construct compact $K_i \subseteq U_i$ with $K \subseteq \cup_i K_i$, which follows from splittability.

Now we show that open sets are $\mu^*$-measurable. Let $U$ be open and $A\subseteq X$ be arbitrary. We want to show that for any open $V \supseteq A$, we have $\mu(V) \ge \mu^*(A\cap U) + \mu^*(A\cap U^c)$, so we just need to show that $\mu(V) \ge \mu(V\cap U) + \mu^*(V\setminus U)$. For any compact $K \subseteq V\cap U$, let $W = V \setminus K$, then for any compact $L \subseteq W$ we have $\mu(V) \ge \lambda(K\cup L) = \lambda(K) + \lambda(L)$, so $\mu(V) \ge \lambda(K) + \mu(W) \ge \lambda(K) + \mu^*(V\setminus U)$, so $\mu(V) \ge \mu(V\cap U) + \mu^*(V\setminus U)$.
\end{proof}

\begin{cor} Every regular content on a locally compact Hausdorff space extends to a unique inner Radon measure. There is a bijection between Radon measures and inner Radon measures on such spaces.
\end{cor}
\begin{proof} Suppose $\lambda$ is a regular content, and let $\mu$ be the associated Radon measure. Define $\mu_{in}$ on Borel sets $E$ by $\mu_{in}(E) = \sup \{\mu(K) \mid K \subseteq E,\ K\text{ \rm compact}\}$. To show $\mu_{in}$ is a Radon inner measure, we just need to check it is countably additive. Let $E = \bigcup_i E_i$ with $E_i \cap E_j = \emptyset$ for $i \ne j$. We clearly have $\mu_{in}(E) \ge \sum_i \mu_{in}(E_i)$. For the other direction, if $K \subseteq E$ is compact, then $\mu(K) < \infty$ implies that $\mu(K \cap E_i) = \mu_{in}(K \cap E_i)$ for all $i$ by Proposition \ref{radon-sigma-finite}, so $\mu_{in}(K) = \mu(K) = \sum_i \mu(K \cap E_i) = \sum_i \mu_{in}(K \cap E_i)$.

The uniqueness of the extension of $\lambda$ and the correspondence between Radon measures and inner Radon measures will both follow if we show that any Radon inner measure $\nu$ is outer regular on compact sets. So suppose $K$ is compact, and let $U$ be any open set which contains $K$ and is contained in a compact set. Then $\nu(U) < \infty$, so for any $\epsilon$ there is a compact set $L \subseteq U\setminus K$ such that $\nu(U\setminus K) \le \nu(L)+\epsilon$. Then $U\setminus L$ is an open set which contains $K$ and has $\nu(U\setminus L) < \nu(K) + \epsilon$.
\end{proof}

\begin{ex} Consider the product topology on $\RR\times X$, where $X$ is an uncountable set with the discrete topology. Let $\lambda$ be the natural content on finite unions of closed intervals in copies of $\RR$, and let $\mu, \mu_{in}$ be the associated Radon measure and Radon inner measure. Then the set $\{0\} \times X$ is not $\sigma$-finite with respect to $\mu$, but has $\mu_{in}$-measure $0$.
\end{ex}

\begin{defn} If $\mu$ is a Borel measure, define its \emph{support} to be the set of points $p$ such that $p \in U$, $U$ open imply $\mu(U) \ne 0$.
\end{defn}

\begin{prop} The support of a Borel measure on a topological space $X$ is always closed. If $\mu$ is inner regular on open sets, then $\mu(X \setminus \supp \mu) = 0$.
\end{prop}

\begin{defn} If $\mu$ is a Borel measure, then a family $\cD$ of disjoint nonempty compact sets is called a \emph{concassage} for $\mu$ if
\begin{itemize}
\item for any $K \in \cD$ and any $U$ open with $K \cap U \ne \emptyset$ we have $\mu(K \cap D) > 0$, and

\item for any Borel set $E$, we have $\mu(E) = \sum_{K \in \cD} \mu(E \cap K)$.
\end{itemize}
\end{defn}

\begin{prop} Every inner Radon measure $\mu$ on a locally compact Hausdorff space has a concassage.
\end{prop}
\begin{proof} Let $\cD$ be any maximal disjoint collection of nonempty disjoint compact sets $K$ such that the restriction of $\mu$ to $K$ has full support (such $\cD$ exists by Zorn's Lemma). First, suppose for contradiction that there is a Borel set $E$ such that $E \cap \bigcup_{K \in \cD} K = \emptyset$ but $\mu(E) > 0$. Then by inner regularity we may assume that $E$ is compact, and then by considering the support of the restriction of $\mu$ to $E$ we see that $\cD$ is not maximal.

Now let $C$ be any compact set, and let $U$ be an open set containing $C$ which is contained in a compact set. Then from $\mu(U) < \infty$, we see that $U$ can intersect at most countably many sets $K$ in $\cD$, so the same is true for $C$, and we have $\mu(C) = \mu(C \cap \bigcup_{K \in \cD} K) = \sum_{K \in \cD} \mu(C \cap K)$. By inner regularity, this implies that for any Borel set $E$, we have $\mu(E) \le \sum_{K \in \cD} \mu(E \cap K)$, and the other inequality follows from disjointness of the sets in $\cD$.
\end{proof}

\begin{defn} If $X,Y$ are locally compact Hausdorff spaces with Radon measures $\mu,\nu$, then we define the \emph{Radon product measure} $\mu\widehat{\times}\nu$ on $X\times Y$ to be the unique Radon measure such that for every open subset $U$ of $X\times Y$, we have $\mu\widehat{\times}\nu(U) = \sup_{K \in \cK} \mu\times\nu(K)$, where $\cK$ is the collection of finite unions of products of compact subsets of $X$ and $Y$. (Note that if $\mu, \nu$ are $\sigma$-finite, then the restriction of $\mu\widehat{\times}\nu$ to the product $\sigma$-algebra on $X\times Y$ is $\mu\times\nu$.)
\end{defn}

\begin{prop}\label{radon-product-open} Suppose that $(X,\mu)$ is a $\sigma$-finite Radon measure space, $Y$ is a topological space, and that $U$ is an open subset of $X \times Y$. Then for any $r \in \RR$, the set of $y \in Y$ such that $\mu(U_y) > r$ is an open subset of $Y$. In fact, for any $y$ with $\mu(U_y) > r$, there are open sets $V \subseteq X$ and $W \subseteq Y$ such that $\mu(V) > r$, $y \in W$, and $V\times W \subseteq U$.
\end{prop}
\begin{proof} Suppose that $\mu(U_y) > r$. By inner regularity, there is some compact set $K \subseteq U_y$ with $\mu(K) > r$. Now cover the compact set $K\times \{y\}$ with finitely many open rectangles contained in $U$.
\end{proof}

\begin{lem}[Weak Fubini for Radon Products]\label{weak-fubini-radon} Let $(X,\mu)$ and $(Y,\nu)$ be $\sigma$-finite Radon spaces, and suppose that $E \subseteq X\times Y$ is a countable intersection of open subsets of $X\times Y$. If $\nu(E_x) = 0$ for $\mu$-a.e. $x \in X$, then for $\nu$-a.e. $y \in Y$ we have $\mu(E_y) = 0$.
\end{lem}
\begin{proof} We may assume WLOG that $\mu(X) = \nu(Y) = 1$. Suppose for contradiction that there is some $\epsilon > 0$ such that the set of $y \in Y$ with $\mu(E_y) > \epsilon$ has $\nu$-measure greater than $\epsilon$. Write $E = \cap_n U_n$ with $U_1 \supseteq U_2 \supseteq \cdots$ open subsets of $X\times Y$. Then by the previous Proposition for each $n$ there is a compact set $K_n \subseteq Y$ with $\nu(K_n) > \epsilon$, and such that for all $y \in K_n$ we have $\mu((U_n)_y) > \epsilon$, and there is a set $V_n \subseteq U_n$ which is a finite union of open rectangles, such that for all $y \in K_n$ we have $\mu((V_n)_y) > \epsilon$. Since $\mu\times\nu(V_n) > \epsilon^2$, we see that the set of $x$ such that $\nu((V_n)_x) > \epsilon^2/2$ has $\mu$-measure at least $\epsilon^2/2$, and thus the same is true if we replace $V_n$ by $U_n$. Taking a decreasing limit of measurable subsets of $X$, we see that the $\mu$-measure of the set of $x$ such that $\nu(E_x) > \epsilon^2/2$ is at least $\epsilon^2/2$, a contradiction.
\end{proof}

\begin{defn} If $G$ is a locally compact Hausdorff group and $\mu$ is a Borel measure on $G$, then $\mu$ is a \emph{left Haar measure} on $G$ if $\mu(gE) = \mu(E)$ for $g \in G$ and $E$ Borel, and $\mu$ is Radon.
\end{defn}

\begin{thm}[Haar measure] If $G$ is a locally compact Hausdorff group, then there is a unique (up to scale) left Haar measure on $G$.
\end{thm}
\begin{proof} For $K$ compact and $V$ with nonempty interior, let $(K:V)$ be the minimum number of left translates of $V$ that are needed to cover $K$. Pick $K_0$ compact with nonempty interior. For every $U$, define $\mu_U$ on compact sets by
\[
\mu_U(K) = \frac{(K:U)}{(K_0:U)}.
\]
Then for all $K,U$ we have $0 \le \mu_U(K) \le (K:K_0)$. We consider each $\mu_U$ as a point in $\prod_K [0,(K:K_0)]$. For each open $V$, let $C(V)$ be the closure of the set of $\mu_U$s with $U \subseteq V$. By compactness, there exists $\mu \in \cap_V C(V)$. For $K_1, K_2$ disjoint, find $V$ open such that $K_1V^{-1} \cap K_2V^{-1} = \emptyset$, then from $\mu \in C(V)$ we see that $\mu(K_1 \cup K_2) = \mu(K_1) + \mu(K_2)$. Thus $\mu$ defines a left-invariant content on the compact sets of $G$, so there is a left-invariant Radon measure on $G$ by Theorem \ref{content-measure}.

%To prove uniqueness, suppose $\mu, \nu$ are left Haar measures and $K,L$ are compact, $L$ with nonempty interior. Since $x \mapsto \frac{1}{\mu(Lx)}$ is integrable on compact sets (to see this, note for any continuous $g$ supported in $L$, the function $x \mapsto \int_G g(tx) d\mu(t)$ is continuous in $x$), then by a version of Fubini and left-invariance we have
%\begin{align*}
%\nu(K) &= \int_G \int_G \frac{1_{x\in K, yx \in L}}{\mu(Lx^{-1})} d\mu(y) d\nu(x)\\
%&= \int_G \int_G \frac{1_{y^{-1}x\in K, x \in L}}{\mu(Lx^{-1}y)} d\mu(y) d\nu(x)\\
%&= \int_G \int_G \frac{1_{y^{-1} \in K, x \in L}}{\mu(Ly)} d\mu(y) d\nu(x)\\
%&= \nu(L) \int_G \frac{1_{y^{-1} \in K}}{\mu(Ly)} d\mu(y),
%\end{align*}
%so $\nu(K)/\nu(L)$ does not depend on $\nu$.

%If $G$ is $\sigma$-compact, then we can prove uniqueness as follows instead: the Radon-Nikodym derivative $h = \frac{d\mu}{d(\mu+\nu)}$ is left-invariant up to null sets, so by Fubini applied to $\int_{G\times G} |h(gx)-h(x)|\ d(g,x)$, we see that there is some $x$ such that $h(gx) = h(x)$ for almost all $g$, so $\mu$ and $\nu$ differ by a constant factor.

%Alternate proof of uniqueness: 
%To prove uniqueness, suppose $\mu, \nu$ are left Haar measures and $K,L$ are compact, $L$ with nonempty interior (so $\mu(L), \nu(L) > 0$). Let $G_0$ be the subgroup generated by $K, L$, so that $G_0$ is $\sigma$-compact (as well as clopen and locally compact Hausdorff) - we will show that the restriction of $\mu, \nu$ to $G_0$ differ by a constant factor. The Radon-Nikodym derivative $h = \frac{d\mu}{d(\mu+\nu)}$ is left-invariant up to null sets, so by Fubini applied to $\int_{G_0\times G_0} |h(gx)-h(x)|\ d(g,x)$, we see that there is some $x$ such that $h(gx) = h(x)$ for almost all $g \in G_0$, so $\mu|_{G_0}$ and $\nu|_{G_0}$ differ by a constant factor. Thus $\nu(K)/\nu(L) = \mu(K)/\mu(L)$. (Problem: $(g,x) \mapsto gx$ is not measurable w.r.t. to the $\sigma$-algebra generated by arbitrary boxes when $|G_0| > 2^{\aleph_0}$. What do?)

To prove uniqueness, suppose $\mu, \nu$ are left Haar measures and $K,L$ are compact, $L$ with nonempty interior (so $\mu(L), \nu(L) > 0$). Let $G_0$ be the subgroup generated by $K, L$, so that $G_0$ is $\sigma$-compact (as well as clopen and locally compact Hausdorff), and restrict everything to $G_0$. Suppose for contradiction that $\nu(K)/\nu(L) \ne \mu(K)/\mu(L)$, and rescale $\mu,\nu$ so that $\mu(K) - \nu(K)$ and $\mu(L) - \nu(L)$ have opposite signs. For every $\epsilon > 0$, let $(P_\epsilon,N_\epsilon)$ be a Hahn decomposition (Theorem \ref{hahn-decomposition}) for $\mu-(1+\epsilon)\nu$ restricted to $G_0$, and let $N = \cap_{n > 0} N_{1/n}$. Then for any $g \in G_0$, $gN\setminus N = \cup_{n > 0}\ gN\cap P_{1/n}$ is a null set with respect to $\mu+\nu$. Now consider the set $\{(g,x) \in G_0\times G_0 \mid x \in N \iff x \not\in gN\}$ - this is a Borel subset of $G_0 \times G_0$ such that every column is null, and to finish we just need to show that some row is null, which can be shown using Lemma \ref{weak-fubini-radon}: we choose $N_0 \subseteq N \subseteq N_1$ such that $N_1$ and $G_0 \setminus N_0$ are countable intersections of open subsets of $G_0$ with $N_1\setminus N_0$ null, so $\{(g,x) \mid g^{-1}x \in N_1, x \not\in N_0\}$ is a countable intersection of open subsets of $G_0 \times G_0$, etc.
\end{proof}

\begin{prop} Let $G$ be a locally compact Hausdorff group with left Haar measure $\mu$, let $K$ be a compact subset of $G$ with nonempty interior, and let $G_0$ be the clopen, $\sigma$-finite subgroup of $G$ generated by $K$. Then a Borel subset $E$ of $G$ is $\sigma$-finite iff it intersects at most countably many left cosets of $G_0$, and if so we have $\mu(E) = \sum_{h \in G/G_0} \mu(E\cap hG_0)$.

The corresponding inner Radon measure $\mu_{in}$ decomposes: for any Borel set $E$, we have $\mu_{in}(E) = \sum_{h \in G/G_0} \mu(E\cap hG_0)$.
\end{prop}

\begin{defn} If $G$ is a locally compact Hausdorff group, then the \emph{modular function} $\Delta : G \rightarrow \RR^+$ is defined by $\mu(Kg) = \Delta(g)\mu(K)$, where $\mu$ is a left Haar-measure on $G$. If $\Delta(G) = \{1\}$, then $G$ is called \emph{unimodular}.
\end{defn}

\begin{prop} The modular function is continuous.
\end{prop}
\begin{proof} Fix $\epsilon > 0$. Let $K$ be a compact set with nonempty interior, and let $U \supseteq K$ be open such that $\mu(K) \le \mu(U) < (1+\epsilon)\mu(K)$. By compactness, there is a neighborhood of the identity $V$ with $KV \subseteq U$. For $g \in V$, we have $\Delta(g) = \frac{\mu(Kg)}{\mu(K)} \le \frac{\mu(U)}{\mu(K)} < 1 + \epsilon$, and for $g \in V^{-1}$ we have $\Delta(g) = \frac{\mu(Ug)}{\mu(U)} \ge \frac{\mu(K)}{\mu(U)} > 1-\epsilon$.
\end{proof}

\begin{prop}\label{big-difference} If $G$ is a locally compact Hausdorff group with left Haar measure $\mu$ and $\mu_{in}(A) > 0$, then $AA^{-1}$ is a neighborhood of the identity.
\end{prop}
\begin{proof} Choose $K \subseteq A$ compact with $\mu(K) > 0$, choose $U \supseteq K$ open such that $\mu(U) < 2\mu(K)$, and find $V$ a neighborhood of the identity such that $VK \subseteq U$. Then for any $g \in V$, we have $\mu(gK\cup K) \le \mu(VK) \le \mu(U) < 2\mu(K)$, so $gK \cap K \ne \emptyset$.
\end{proof}

Next we'll use this to prove that measurable homomorphisms of locally compact Hausdorff groups are continuous, following \cite{kleppner-measurable}. (An even stronger statement is proved there.)

\begin{lem}\label{asoo} For $N$ a subgroup of $G$, TFAE:
\begin{enumerate}
\item for all $x \in G$, $[N:xNx^{-1}\cap N] \le \aleph_0$,

\item each double coset $NxN$ is a union of countably many left $N$ cosets,

\item for each $x$ there is a countable set $D$ such that $Nx \subseteq DN$,

\item if $C \subseteq G$ is countable, and $M$ is the subgroup generated by $N \cup C$, then $[M:N] \le \aleph_0$.
\end{enumerate}
\end{lem}
\begin{proof} For $(1) \iff (2)$, the double coset $NxN$ is the orbit of $xN \in G/N$ under left translation by $N$, and the stabilizer is $xNx^{-1} \cap N$. $(2) \iff (3)$ is obvious, and $(3) \implies (4), (4) \implies (2)$ are easy.
\end{proof}

\begin{defn} A subgroup $N$ of $G$ is called \emph{asoo} if it satisfies the equivalent conditions of Lemma \ref{asoo}.
\end{defn}

\begin{prop} Countable subgroups and normal subgroups are asoo. Any open $\sigma$-compact subgroup of a topological group is asoo. If $\phi : G \rightarrow H$ is a homomorphism and $L$ is asoo in $H$ then $\phi^{-1}(L)$ is asoo in $G$. If $\phi$ is onto and $N$ is asoo in $G$, then $\phi(N)$ is asoo in $H$. If $N$ is asoo and $C$ is countable, then the subgroup generated by $N \cup C$ is asoo.
\end{prop}

\begin{prop} If $G$ is $\sigma$-compact and $U_n$ is a countable family of neighborhoods of the identity, then there is a compact normal subgroup $K$ of $G$ such that $K \subseteq \bigcap_n U_n$ and $G/K$ is separable.
\end{prop}
\begin{proof} Let $G = \bigcup_n F_n$, with $F_n$ an increasing sequence of compact subsets of $G$. Let $V_0$ be a compact neighborhood of the identity. For each $n$, we can find a symmetric neighborhood of the identity $V_{n+1}$ such that $V_{n+1}^2 \subseteq V_n \cap U_n$ and for all $x \in F_n$ we have $xV_{n+1}x^{-1} \subseteq V_n$. Take $K = \bigcap_n V_n$. To finish, we need to show that for any open $W$ containing the identity, there is some $n$ such that $V_n \subseteq WK$. Otherwise, each $V_n \setminus WK$ is a compact nonempty subset of $V_0$, so by the finite intersection property $K \setminus WK$ is nonempty, contradiction.
\end{proof}

\begin{lem} Let $G$ be a locally compact Hausdorff group which is either separable or $\sigma$-compact and $N$ a null asoo subgroup of $G$. Then there is a nonmeasurable set $S \subseteq G$ with $S = NS$.
\end{lem}
\begin{proof} Let $\mu$ be left Haar measure. First, assume $G$ is separable, and let $U_1 \supseteq U_2 \supseteq \cdots$ be a basis of neighborhoods of the identity. For each $n$ take $x_n \in U_n \setminus N$ (which is nonempty since $\mu(U_n) > 0$). Let $M$ be generated by $N$ and the $x_n$s, and let $Y$ be a set of right coset representatives of $M$ in $G$. Take $S = NY$. Suppose $S$ measurable, and let $X$ be a set of left coset representatives of $N$ in $M$. Since $X$ is countable and $G = MY = XNY = XS$, we have $\mu_{in}(S) > 0$, so by Proposition \ref{big-difference} there is an $n$ with $U_n \subseteq SS^{-1}$. But then $x_n \in SS^{-1}$, so $S\cap x_nS \ne \emptyset$, so $N\cap x_nN \ne \emptyset$, contradiction.

Next, assume that $N$ is not closed, and take $x \in \overline{N}\setminus N$. Let $M$ be generated by $N$ and $x$, and define $Y,S,X$ as before. If $S$ is measurable, then as above we see there is an open neighborhood of the identity $U \subseteq SS^{-1}$. Since $x \in \overline{N}$, we have $xN \cap U \ne \emptyset$, so $S \cap xNS \ne \emptyset$, so $N \cap xN \ne \emptyset$, contradiction.

Finally, suppose that $N$ is closed and $G$ is $\sigma$-compact. Then there is a compact normal subgroup $K$ of $G$ such that $G/K$ is separable. Let $\phi$ be the quotient map. If $\phi(N)$ is null in $G/K$, then the first case lets us finish. Otherwise, since $\phi(N)$ is a subgroup of $G/K$, Proposition \ref{big-difference} shows $\phi(N)$ is open, so $NK$ is open in $G$, so $[NK:N] > \aleph_0$. Let $C$ be a countable subset of $NK$ with infinite image in $NK/N$, and let $M$ be the subgroup generated by $N \cup C$. If $M$ is not closed, the second part lets us finish. If $M$ is closed, then it is locally compact Hausdorff, and from $[M:N] = \aleph_0$ we see that $N$ has positive inner $M$-Haar measure, so by Proposition \ref{big-difference} $N$ is open in $M$. Thus $M/N$ is a discrete closed subset of the compact set $NK/N$, so it is finite, contradiction.
\end{proof}

\begin{thm} Suppose $\phi : G \rightarrow H$ is a homomorphism of locally compact Hausdorff groups such that for every open set $U \subseteq H$, $\phi^{-1}(U)$ is measurable (with respect to the completion of the Haar measure). Then $\phi$ is continuous.
\end{thm}
\begin{proof} We may assume WLOG that $G$ is compactly generated. By Proposition \ref{big-difference} and the fact that every neighborhood $U$ of the identity contains a neighborhood $V$ with $VV^{-1} \subseteq U$, it's enough to show that for every open neighborhood $V$ of the identity, $\phi^{-1}(V)$ is not null. Suppose $\phi^{-1}(V)$ is null, and let $L$ be an open $\sigma$-compact subgroup of $H$. Then $L$ is asoo and is contained in a union of countably many left translates of $V$, so $\phi^{-1}(L)$ is a null asoo subgroup of $G$. By the Lemma, there is a nonmeasurable $S \subseteq G$ with $S = \phi^{-1}(L)S$. We have $L\phi(S)$ open, so by hypothesis $S = \phi^{-1}(L)S = \phi^{-1}(L\phi(S))$ is measurable, contradiction.
\end{proof}

\subsection{Integration}

\begin{defn} If $f:X\rightarrow Y$ and $\cB$ is a $\sigma$-algebra on $Y$, then $\sigma(f)$ is the $\sigma$-algebra on $X$ generated by $f^{-1}(S)$ for $S \in \cB$. We say that $f:(X,\Sigma) \rightarrow (Y,\cB)$ is $\gS$-\emph{measurable}, or just \emph{measurable} if $\gS$ is clear, if $\sigma(f) \subseteq \gS$ (if unspecified, $\cB$ is usually taken to be the Borel sets of $Y$).
\end{defn}

\begin{prop} $f:(X,\gS) \rightarrow [-\infty,\infty]$ is measurable iff $f^{-1}([-\infty,a]) \in \gS$ for all $a \in \RR$. If $f_1, ..., f_n$ are measurable and $g:\RR^n \rightarrow [-\infty,\infty]$ is Borel measurable, then $g(f_1, ..., f_n)$ is measurable. If $f_k$ is a sequence of measurable functions, then $\sup f_k$ is measurable.
\end{prop}

\begin{prop} If $f_k : X \rightarrow Y$ is a sequence of measurable functions to a metric space and $f_k \rightarrow f$ pointwise, then $f$ is measurable.
\end{prop}
\begin{proof} For any open set $U$ the collection of $x \in X$ such that $f_k(x)$ are eventually all in $U$ is measurable, and this set contains $f^{-1}(U)$ and is contained $f^{-1}(\overline{U})$. Since every open set in a metric space is a countable union of open subsets whose closures are contained in it, the preimage of every open set is measurable.
\end{proof}

\begin{defn} A \emph{simple function} is a function which can be written as a finite linear combination of measurable sets. Equivalently, a function is simple if it is measurable and has a finite range.
\end{defn}

\begin{defn} For $f \ge 0$ measurable (up to a null set), we define the \emph{integral} of $f$ with respect to a measure $\mu:\gS\rightarrow [0,\infty]$ to be
\[
\int f\ d\mu = \sup\Big\{\sum_{i = 1}^k c_i\mu(A_i) \mid c_1, ..., c_k \ge 0,\ A_1, ..., A_k \in \gS,\ \sum_{i=1}^k c_i\cdot 1_{x \in A_i} \le f(x)\Big\}.
\]
A measurable (up to a null set) complex-valued function $f$ is \emph{integrable} if $\int |f| d\mu < \infty$. We extend the integral to all integrable functions by linearity.
\end{defn}

\begin{thm}[Markov's Inequality]\label{markov} For $t > 0$, $\mu(\{|f| \ge t\}) \le \frac{1}{t}\int |f|\ d\mu$.
\end{thm}

\begin{prop} For $f,g \ge 0$ measurable, we have $\int f+g\ d\mu = \int f\ d\mu + \int g\ d\mu$.
\end{prop}
\begin{proof} For any finite $S \subset [0,\infty]$, define $f_S$ by
\[
f_S(x) = \max \{s \in S \mid s \le f(x)\}.
\]
Note $f_S$ is a simple function and $\int f\ d\mu = \sup_S \int f_S\ d\mu$. For any $S$ and any $n$, if we let $S_n = \{\frac{k}{n}s\mid k \le n,\ s\in S\}$, then $(f+g)_S \le \frac{n-1}{n}(f_{S_n}+g_{S_n})$.
\end{proof}

\begin{prop} Any Riemann integrable function $f:[0,1] \rightarrow \CC$ is Lebesgue integrable, with the same integral.
\end{prop}

\begin{prop} If $f:X \rightarrow [0,\infty]$ is measurable, then $\{(x,t) \mid 0 \le t \le f(x)\}$ is measurable in $X\times [0,\infty]$, with $\mu\times\gl$-measure $\int_X f\ d\mu = \int_0^\infty \mu(\{x \mid f(x) \ge t\})\ dt$.
\end{prop}
\begin{proof} For any $c > 1$, if we round positive values of $f$ up or down to the nearest $c^n$, $n \in \ZZ$, we see that the product outer measure of $\{(x,t) \mid 0 \le t \le f(x)\}$ is at most $c$ times $\int_X f\ d\mu$.
\end{proof}

\begin{thm}[Monotone Convergence Theorem]\label{monotone-convergence} If $f_k$ is a sequence of measurable functions with $0 \le f_k \le f_{k+1}$ for all $k$ and $f$ is the pointwise limit of the $f_k$, then $f$ is measurable and $\int f\ d\mu = \lim_k \int f_k\ d\mu$.
\end{thm}
\begin{proof} It's enough to prove this when $f$ is the characteristic function of a measurable set $A$. Fix $\epsilon > 0$, and for each $k$ set $A_k = \{x \mid f_k(x) \ge 1-\epsilon\}$, then from $\cup_k A_k = A$, we have $\lim_k \mu(A_k) = \mu(A)$, so $\lim_k \int f_k\ d\mu \ge (1-\epsilon)\mu(A)$.
\end{proof}

\begin{lem}[Fatou's Lemma] If $f_k \ge 0$ are measurable, then $\int \liminf_k f_k\ d\mu \le \liminf_k \int f_k\ d\mu$.
\end{lem}
\begin{proof} $\int \liminf_k f_k\ d\mu = \lim_k \int \inf_{l\ge k} f_l\ d\mu \le \liminf_k \int f_k\ d\mu$.
\end{proof}

\begin{cor} If $f_k$ measurable, $|f_k| \le g$, $g$ integrable, then
\[
\int \liminf f_k\ d\mu \le \liminf \int f_k\ d\mu \le \limsup \int f_k\ d\mu \le \int \limsup f_k\ d\mu.
\]
\end{cor}

\begin{thm}[Dominated Convergence Theorem]\label{dominated-convergence} If $f_k$ measurable, $|f_k| \le g$, $g$ integrable, $f_k \rightarrow f$ pointwise, then $\lim_k \int f_k\ d\mu = \int f\ d\mu$, and $\lim_k \int |f_k - f|\ d\mu = 0$.
\end{thm}

\begin{thm}[Jensen]\label{jensen} If $\mu(X) = 1$, $g$ real $\mu$-integrable, $\varphi$ convex, then $\varphi(\int_X g\ d\mu) \le \int_X \varphi \circ g\ d\mu$.
\end{thm}
\begin{proof} Let $x_0 = \int_X g\ d\mu$. Since $\varphi$ is convex, there are $a,b \in \RR$ such that $ax+b \le \varphi(x)$ and $ax_0+b = \varphi(x_0)$. Integrating both sides of $ag(t) + b \le \varphi(g(t))$ gives the inequality.
\end{proof}

\begin{thm}[Radon-Nikodym Theorem]\label{radon-nikodym} If $\mu, \nu$ are $\sigma$-finite measures on $X$ ($\nu$ possibly signed or complex) and $\nu \ll \mu$, then there exists a measurable function $f$ (unique up to a $\mu$-null set) such that for any measurable set $A$, $\nu(A) = \int_A f\ d\mu$.
\end{thm}
\begin{proof} We just need to prove this in the positive, finite case. Let $\cF$ be the family of measurable functions $f$ such that for all measurable $A$, $\nu(A) \ge \int_A f\ d\mu$. Note that $\cF$ is closed under maximum, and by the Monotone Convergence Theorem \ref{monotone-convergence} $\cF$ is closed under countable monotone limits, so there is some $f \in \cF$ with $\int_X f\ d\mu = \sup_{g \in \cF} \int_X g\ d\mu$. Let $\nu_0 = \nu - \int f\ d\mu$. If $\nu_0(X) > 0$, take $\epsilon > 0$ such that $\nu_0(X) > \epsilon \mu(X)$, and let $(N,P)$ be a Hahn decomposition \ref{hahn-decomposition} of $\nu_0 - \epsilon \mu$. But then $f + \epsilon\cdot 1_P \in \cF$ and $\mu(P) > 0$, contradicting our choice of $f$.
\end{proof}

\begin{defn} If $\mu, \nu$ have $\nu = \int f\ d\mu$, then the \emph{Radon-Nikodym derivative} $\frac{d\nu}{d\mu}$ is defined to be the equivalence class of $f$ when we quotient by $\mu$-null functions.
\end{defn}

\begin{prop} If $\mu$ is a complex measure, then $\frac{d\mu}{d|\mu|}$ has absolute value $1$ $|\mu|$-almost everywhere.
\end{prop}
\begin{proof} Let $f$ be a representative of $\frac{d\mu}{d|\mu|}$. For any $\epsilon > 0$, the set where $|f| < 1-\epsilon$ has measure $0$, since otherwise its $|\mu|$ measure would be smaller than itself by a factor of $1-\epsilon$. By dividing up the set where $|f| > 1+\epsilon$ into $O(\tfrac{1}{\sqrt{\epsilon}})$ many subsets based on the argument of $f$, we see that it must also have measure $0$, since otherwise its $|\mu|$ measure would be larger than itself by a factor of $1 + \frac{\epsilon}{2}$.
\end{proof}

\begin{prop} Where the relevant Radon-Nikodym derivatives make sense, we have $\frac{d(\nu+\mu)}{d\lambda} = \frac{d\mu}{d\lambda} + \frac{d\nu}{d\lambda}$, $\frac{d\nu}{d\lambda} = \frac{d\nu}{d\mu} \frac{d\mu}{d\lambda}$, $\frac{d|\nu|}{d\mu} = |\frac{d\nu}{d\mu}|$, and $\int g\ d\mu = \int g\frac{d\mu}{d\lambda}\ d\lambda$.
\end{prop}

\begin{prop}\label{tonelli-indicator} If $E \subseteq X\times Y$ is measurable and $\mu\times\nu(E) < \infty$, then for $\mu$-almost every $x \in X$ $E_x$ is measurable up to a $\nu$-null set, the function $g(x) = \mu(E_x)$ is measurable up to a $\mu$-null set, and $\int g\ d\mu = \mu\times\nu(E)$.
\end{prop}
\begin{proof} By definition of $\mu\times\nu$, there is an $F \supseteq E$ which is a countable decreasing intersection of countable unions of measurable rectangles, such that $\mu\times\nu(E) = \mu\times\nu(F)$. Since $\mu\times\nu(E) < \infty$, $F\setminus E$ is $\mu\times\nu$-null, so we may replace $E$ by $F$ without changing $g$ (aside from on a $\mu$-null set) by Proposition \ref{product-null} and then apply monotone \ref{monotone-convergence} and dominated \ref{dominated-convergence} convergence to reduce to the case of a finite union of measurable rectangles.
\end{proof}

\begin{thm}[Fubini's Theorem]\label{fubini} If $\int_{X\times Y} |f(x,y)|\ d(x,y) < \infty$, where $d(x,y)$ is the maximal product measure on $X\times Y$, then for a.e. $x\in X$ $f(x,y)$ is integrable in $y$, and we have $\int_{X\times Y} f(x,y)\ d(x,y) = \int_X \int_Y f(x,y)\ dy\ dx$.
\end{thm}

\begin{thm}[Tonelli's Theorem]\label{tonelli} If $X,Y$ are $\sigma$-finite, then $\int_{X\times Y} |f(x,y)|\ d(x,y) = \int_X \int_Y |f(x,y)|\ dy\ dx$.
\end{thm}
\begin{proof} Assume $f \ge 0$. The assumptions of either Fubini or Tonelli imply that $f$ can be written as the pointwise limit of an increasing sequence $\phi_n$ of nonnegative simple functions that each vanish outside a set of finite measure. Thus, using Proposition \ref{tonelli-indicator}, for almost every fixed $x$ the function $y\mapsto f(x,y) = \lim_n \phi_n(x,y)$ is measurable up to a null set, and by monotone convergence \ref{monotone-convergence} the function $x \mapsto \int_Y f(x,y)\ dy = \lim_n \int_Y \phi_n(x,y)\ dy$ is measurable up to a null set. Applying monotone convergence and Proposition \ref{tonelli-indicator} again, we get
\begin{align*}
&\int_X\int_Yf(x,y)\ dy\ dx = \lim_n \int_X\int_Y\phi_n(x,y)\ dy\ dx\\
&= \lim_n \int_{X\times Y}\phi_n(x,y)\ d(x,y) = \int_{X\times Y}f(x,y)\ d(x,y).\qedhere
\end{align*}
\end{proof}

A lot of the next bits are from \cite{folland}.

\begin{prop} If $X,Y$ are locally compact Hausdorff with Radon measures $\mu, \nu$ and $U$ is open in $X\times Y$, then $x \mapsto \nu(U_x)$ is lower semicontinuous and $\mu{\widehat\times}\nu(U) = \int \nu(U_x)\ d\mu(x)$.
\end{prop}
\begin{proof} This follows directly from Proposition \ref{radon-product-open}, the definition of the integral, and the fact that Radon measures are inner regular on open sets.
\end{proof}

\begin{thm}[Fubini-Tonelli for Radon Products] If $\mu,\nu$ are $\sigma$-finite Radon measures on locally compact Hausdorff spaces $X,Y$, $f$ is Borel measurable on $X\times Y$, and either $f \ge 0$ or $|f|$ is integrable, then $\int_{X\times Y} f\ d\mu{\widehat\times}\nu = \int_X \int_Y f\ d\nu\ d\mu$.
\end{thm}

\begin{prop} If $X,Y$ are locally compact Hausdorff, then every $f \in C_c(X\times Y)$ is measurable with respect to the product of the Borel $\sigma$-algebras.
\end{prop}
\begin{proof} Follows from Stone-Weierstrauss, Urysohn's Lemma, and the fact that pointwise limits of measurable functions are measurable.% TODO: state Urysohn's lemma, fill this out
\end{proof}

\begin{defn} Write $f \prec U$ if $0 \le f \le \chi_U$ and $\supp(f) \subseteq U$.
\end{defn}

\begin{thm}[Riesz Representation Theorem] If $X$ is a locally compact Hausdorff space and $I$ is a positive linear functional on $C_c(X)$, then there is a unique Radon measure $\mu$ such that $I(f) = \int f\ d\mu$ for all $f \in C_c(X)$. This $\mu$ satisfies $\mu(K) = \inf\{I(f) \mid f \ge \chi_K\}$ for all compact $K$ and $\mu(U) = \sup\{I(f) \mid f \prec U\}$ for all open $U$.
\end{thm}
\begin{proof} Uniqueness and the formulas for $\mu(U), \mu(K)$ follow from Urysohn's Lemma. For existence, we need to check that the formula for $\mu(K)$ defines a regular content and that the formula $I(f) = \int f\ d\mu$ holds. For the regularity, note that if $K$ is compact and $f \ge \chi_K$, then if we let $U_\epsilon = \{x \mid f(x) > 1 - \epsilon\}$, then for any $g \prec U_\epsilon$ we have $I(g) \le I(f)/(1-\epsilon)$, so $\mu(U_\epsilon) \le I(f)/(1-\epsilon)$.

For the integral formula, if $f \le 1$, for any $N$ we define $K_j = \{x \mid f(x) \ge j/N\}$ and $K_0 = \supp(f)$, and $f_j = \min((f - \frac{j-1}{N})_+, \frac{1}{N})$, so $f = \sum f_j$ and $\chi_{K_j} \le Nf_j \le \chi_{K_{j-1}}$. Then we have $\mu(K_j) \le N\int f_j\ d\mu \le \mu(K_{j-1})$ and $\mu(K_j) \le NI(f_j) \le \mu(K_{j-1})$ (last inequality using outer regularity), so $|I(f) - \int f\ d\mu| \le \mu(K_0)/N$.
\end{proof}

\begin{lem} If $X$ is locally compact Hausdorff, then every bounded real linear functional $I$ on $C_0(X)$ can be written as the difference between two positive linear functionals.
\end{lem}
\begin{proof} For $f \ge 0$ in $C_0(X)$, set $I^+(f) = \sup\{I(g) \mid 0 \le g \le f\}$ (this is finite since $I$ is bounded) and $I^- = I^+ - I$. The only tricky bit is to check that $I^+(f_1 + f_2) \le I^+(f_1) + I^+(f_2)$: if $0 \le g \le f_1 + f_2$, then set $g_1 = \min(g,f_1)$ and $g_2 = g - g_1 = \max(0,g-f_1)$, so $0 \le g_i \le f_i$, so $I(g) \le I^+(f_1) + I^+(f_2)$.
\end{proof}

\begin{thm}[Riesz Representation Theorem for $C_0(X)$] If $X$ is a locally compact Hausdorff space, then $\mu \mapsto I_\mu = (f \mapsto \int f\ d\mu)$ is an isometric isomorphism between complex Radon measures on $X$ under the total variation norm $\|\mu\| = |\mu|(X)$ and bounded linear functionals on $C_0(X)$ under the operator norm $\|I\| = \sup\{|I(f)| \mid \sup_x |f(x)| = 1\}$.
\end{thm}
\begin{proof} Apply Lusin's Theorem (below) to $\frac{d\mu}{d|\mu|}$ to show that $\|\mu\| \le \|I_\mu\|$; the other inequality is easy.
\end{proof}


\subsubsection{Convergence in Measure}

\begin{defn} A sequence of measurable functions $f_n$ converges to $f$ \emph{globally in measure} if $\forall \epsilon > 0$, we have $\lim_n \mu(\{x \mid |f(x) - f_n(x)| \ge \epsilon\}) = 0$, and $f_n \rightarrow f$ \emph{locally in measure} if $\forall \epsilon > 0$ and for all $F \in \Sigma$ with $\mu(F) < \infty$ we have $\lim_n \mu(\{x \in F \mid |f(x) - f_n(x)| \ge \epsilon\}) = 0$.
\end{defn}

\begin{thm}[Riesz] If $f_n \rightarrow f$ globally in measure (or locally in measure on a $\sigma$-finite space) then some subsequence converges to $f$ pointwise almost everywhere.
\end{thm}
\begin{proof} Choose a subsequence $n_k$ such that $\mu(\{x \mid |f(x) - f_{n_k}(x)| \ge \frac{1}{k}\}) < 2^{-k}$.
\end{proof}

\begin{prop} If all subsequences of $f_n$ have a subsequence which converges to $f$ almost everywhere (and $f$ is finite almost everywhere), then $f_n \rightarrow f$ locally in measure.
\end{prop}
\begin{proof} Suppose there is some $F \in \Sigma$ with $\mu(F) < \infty$ and $\epsilon > 0$ such that $\mu(\{x \in F \mid |f(x) - f_n(x)| \ge \epsilon\})$ doesn't converge to $0$. Then there is a $\delta > 0$ and a subsequence $n_k$ such that $\mu(\{x \in F \mid |f(x) - f_{n_k}(x)| \ge \epsilon\}) > \delta$ for all $k$. No such subsequence $f_{n_k}$ can converge almost everywhere to $f$: otherwise, there would be some $K$ such that the set of $x \in F$ with $|f(x) - f_{n_k}(x)| < \epsilon$ for all $k > K$ has measure at least $\mu(F) - \delta$.
\end{proof}

%\begin{prop} If $X = [a,b]$ and $\mu$ is Lebesgue measure, then for any measurable $f:[a,b] \rightarrow \RR$ there is a sequence of step functions converging to $f$ in measure, and similarly for continuous functions.
%\end{prop}

%\begin{prop} In a finite measure space, if $f_n \rightarrow f$ and $g_n \rightarrow g$ in measure, then $f_ng_n\rightarrow fg$ in measure.
%\end{prop}

% TODO: prove these if they are worth proving

\begin{thm}[Egoroff's Theorem] If $M$ is a separable metric space and $f_n$ is a sequence of measurable functions from $A$ to $M$, with $\mu(A) < \infty$, such that $f_n \rightarrow f$ pointwise almost everywhere, then for every $\epsilon > 0$ there is $B \subseteq A$ such that $\mu(B) < \epsilon$ and $f_n \rightarrow f$ uniformly on $A\setminus B$.
\end{thm}
\begin{proof} For every $k$, choose $n_k$ such that $\mu(\{x \in A \mid \exists m > n_k\ d(f(x),f_m(x)) \ge \frac{1}{k}\}) < \frac{\epsilon}{2^k}$ (to see that $x \mapsto d(f(x),f_m(x))$ is measurable, we use separability of $M$).
\end{proof}

\begin{thm}[Lusin's Theorem]\label{lusin} If $f : [a,b] \rightarrow \CC$ is measurable, then $\forall \epsilon > 0$ there exists a compact $E \subseteq [a,b]$ such that $f|_E$ is continuous and $\mu(E) > b-a-\epsilon$. More generally, if $(X,\mu)$ is a Radon measure space and $Y$ is second-countable, and $f : A \rightarrow Y$ is measurable with $\mu(A) < \infty$, then $\forall \epsilon > 0$ there is a compact set $E \subseteq A$ with $\mu(A\setminus E) < \epsilon$ such that $f|_E$ is continuous.
\end{thm}
\begin{proof} (From \cite{feldman-lusin}) Let $U_j$ be an enumeration of a base of open sets for $Y$, and for each $j$ choose $V_j$ open in $X$ such that $f^{-1}(U_j) \subseteq V_j$ and $\mu(V_j \setminus f^{-1}(U_j)) < \frac{\epsilon}{2^j}$. Take $E_1 = A\setminus \bigcup_j (V_j \setminus f^{-1}(U_j))$, so $f^{-1}(U_j) \cap E_1 = V_j\cap E_1$, then let $E$ be a compact set contained in $E_1$ with sufficiently close measure.
\end{proof}


\subsubsection{Lebesgue Integral and Derivatives}

\begin{defn} The \emph{Hardy-Littlewood maximal operator} $M$ takes a locally integrable $f : \RR^n \rightarrow \CC$ to the function $Mf$ given by
\[
Mf(x) = \sup_{r > 0} \frac{\int_{B_r(x)} |f(y)|\ dy}{\lambda(B_r)}.
\]
\end{defn}

\begin{thm}[Weak type Hardy-Littlewood maximal inequality]\label{weak-maximal-inequality} For any integrable function $f : \RR^n \rightarrow \CC$, we have $\lambda(\{Mf > t\}) \le \frac{3^n}{t}\int |f|\ d\lambda$.
\end{thm}
\begin{proof} Let $A_t = \{Mf > t\}$, and let $K$ be any compact set contained in $A_t$. For each $x \in K$, we can find an $r > 0$ such that $\int_{B_r(x)} |f(y)|\ dy > t\lambda(B_r(x))$, and finitely many of these balls $B_r(x)$ cover $K$. Apply the Finite Vitali Covering Lemma \ref{finite-vitali} to get a collection $B_i$ of disjoint balls among these such that $K \subseteq \bigcup_i 3B_i$, then $\lambda(K) \le 3^n\sum_i \lambda(B_i) \le \frac{3^n}{t} \int |f(y)|\ dy$.
\end{proof}

%TODO: Stein's spherical maximal inequality (maybe https://terrytao.wordpress.com/2011/05/21/steins-spherical-maximal-theorem/)

\begin{thm}[Lebesgue Differentiation Theorem]\label{lebesgue-differentiation} If $f : \RR^n \rightarrow \CC$ is locally integrable, then for Lebesgue-a.e. $x$ we have
\[
\lim_{r \rightarrow 0} \frac{\int_{B_r(x)} |f(y)-f(x)|\ dy}{\lambda(B_r)} = 0.
\]
\end{thm}
\begin{proof} First proof: approximate $f$ by a simple function, and apply the Lebesgue Density Theorem \ref{lebesgue-density}.

Second proof: Assume $f$ is supported on a finite ball. By Lusin's Theorem \ref{lusin} and the Tietze Extension Theorem (Corollary \ref{lch-tietze}), we can find $g \in C_c(\RR^n)$ with $\int |f-g|\ d\lambda < \epsilon$. Then
\[
\frac{1}{\lambda(B)}\int_{B} |f(y)-f(x)|\ dy \le \frac{1}{\lambda(B)}\int_B |f(y) - g(y)|\ d\lambda + \frac{1}{\lambda(B)}\int_B |g(y) - g(x)|\ d\lambda + |f(x) - g(x)|.
\]
By Theorem \ref{weak-maximal-inequality} the first summand is at most $t$ away from a set of measure at most $\frac{3^n}{t} \int |f-g|\ d\lambda < \frac{3^n\epsilon}{t}$, and by Markov's inequality the third summand is at most $t$ away from a set of measure at most $\frac{\epsilon}{t}$, while the second summand goes to $0$ as $r$ goes to $0$ since $g \in C_c(\RR^n)$.
\end{proof}

\begin{prop} If $f$ is nondecreasing, then $f$ has only jump discontinuities, and only countably many of them.
\end{prop}

\begin{lem}[Riesz's Rising Sun Lemma]\label{rising-sun} If $U \subseteq \RR$ is open and $g : U \rightarrow \RR$ is continuous, then the set $U_g = \{x \in U \mid \exists y > x \text{ s.t. } (x,y) \subseteq U \text{ and } g(x) < g(y)\}$ is also open, and if $(a,b)$ is a component of $U_g$ then $g(a) \le g(b)$.
\end{lem}

\begin{thm}[Lebesgue]\label{lebesgue-monotone} If $f$ is nondecreasing, then $f$ is differentiable almost everywhere, and $\int_a^b f'(x)\ dx \le f(b) - f(a)$. If $E,Z,I$ are the sets where $f$ is not differentiable, has derivative $0$, and has derivative $\infty$, respectively, then $f(E), f(Z), I$ have measure $0$.
\end{thm}
\begin{proof} (Following \cite{faure-lebesgue}) Set $D^+f(x) = \limsup_{h \downarrow 0} \frac{f(x+h)-f(x)}{h}$, $D_+f(x) = \liminf_{h \downarrow 0} \frac{f(x+h)-f(x)}{h}$, and similarly define $D^-, D_-$ with $h$ approaching $0$ from below.

First we show that if $f$ is continuous and $E$ is any set where $D^+f > u$, then $\lambda^*(f(E)) \ge u\lambda^*(E)$: if $U$ is any open set containing $f(E)$, then $f^{-1}(U)$ is an open set containing $E$, and the rising sun lemma \ref{rising-sun} applied to $g(x) = f(x) - ux$ and $f^{-1}(U)$ shows that $\lambda(U) \ge u\lambda(f^{-1}(U)_g) \ge u\lambda^*(E)$.

Next we show that if $f$ is strictly increasing and $E$ is any set where $D_+ < v$, then $\lambda^*(f(E)) \le v\lambda^*(E)$: let $g(y) = \inf\{z \mid f(z) \ge y\}$ be inverse to $f$, suppose WLOG that no point of discontinuity of $f$ is in $E$, then for any $x \in E$ we have $D^+g(f(x)) > \frac{1}{v}$, and we can reduce to the previous case. We extend this from strictly increasing $f$ to all $f$ by replacing $f$ by $h(x) = f(x)+x$ and noting that $\lambda^*(h(E)) \ge \lambda^*(f(E)) + \lambda^*(E)$ (take any open set containing $h(E)$ and break it into connected components). From this we see that we can drop the continuity assumption in the first case by considering the function $g$ inverse to $f$ again and ignoring the countably many points where either $f$ or $g$ is discontinuous.

Applying the above to $-f(-x)$, we get similar statements for $D^-, D_-$. We'll show that $D^+ \le D_-$ almost everywhere, and similarly $D^- \le D_+$ a.e., so we will have $D^+ \le D_- \le D^- \le D_+ \le D^+$ almost everywhere. Let $E_{uv}$ be the set of $x$ with $D^+f(x) > u > v > D_-f(x)$. Then $u\lambda^*(E_{uv}) \le \lambda^*(f(E_{uv})) \le v\lambda^*(E_{uv})$, so $\lambda^*(E_{uv}), \lambda^*(f(E_{uv}))$ must be $0$.

For the statement about $\int_a^b f'(x)\ dx$, apply Fatou's Lemma to the sequence of functions $f_n(x) = n(f(x+\frac{1}{n}) - f(x))$ (assuming WLOG that $f(x) = f(b)$ for $x > b$).
\end{proof}

\begin{cor} If $f$ has bounded variation, then $f$ is differentiable almost everywhere and $f'$ is Lebesgue integrable.
\end{cor}

\begin{cor} If $f$ is increasing, $\mu_f$ is the Lebsgue-Stieltjes measure \ref{lebesgue-stieltjes}, and $(\mu_f)_{ac}$ is the absolutely continuous part of the Lebesgue decomposition \ref{lebesgue-decomposition} of $\mu_f$ with respect to $\lambda$, then $\frac{d(\mu_f)_{ac}}{d\lambda} = f'$. The singular part $(\mu_f)_s$ can be written as the sum of a discrete measure and some $\mu_c$ with $c$ continuous and $c' = 0$ almost everywhere.
\end{cor}
\begin{proof} Let $g(x) = \int_0^x f'(t)\ dt$, and note that $\mu_g \le \mu_f$ and $\mu_g \ll \lambda$ since $\mu_g(E) = \int_E f'(t)\ dt$ for any Borel set $E$. By the Lebesgue differentiation theorem \ref{lebesgue-differentiation} and the fact that $g'$ exists almost everywhere, we have $g' = f'$ almost everywhere. To finish, we need to check that if $c$ is continuous and $c' = 0$ almost everywhere then $\mu_c \perp \lambda$: if $Z$ is the set where $c' = 0$, then $\mu_c(Z) = \lambda(c(Z)) = 0$.
\end{proof}

\begin{defn} A function $f : I \rightarrow \RR$ is \emph{absolutely continuous} on $I$ if $\forall \epsilon > 0$ $\exists \delta > 0$ such that if $(x_k,y_k) \subseteq I$ are disjoint subintervals with $\sum_k |y_k - x_k| < \delta$ then $\sum_k |f(y_k) - f(x_k)| < \epsilon$.
\end{defn}

\begin{prop} If $f$ is absolutely continuous then $f$ has bounded variation, and the variation of $f$ is also absolutely continuous.
\end{prop}

\begin{thm}[Fundamental Theorem of the Lebesgue Integral]\label{fundamental-lebesgue-integral} A function $f$ is absolutely continuous on $[a,b]$ iff there exists $g$ integrable with $f(x) = f(a) + \int_a^x g(t)\ dt$ for all $x \in [a,b]$. In this case we have $g = f'$ almost everywhere.
\end{thm}
% TODO: give alternate more elementary proof, such as from https://arxiv.org/pdf/1203.1462.pdf

\begin{cor} If $f$ is absolutely continuous and has $f' = 0$ almost everywhere, then $f$ is constant.
\end{cor}
\begin{proof} Direct proof based on Vitali Covering Theorem \ref{vitali-hausdorff}: Let $\mathcal{V}$ be the family of intervals $[x,y] \subseteq [a,b]$ such that $f'(x) = 0$ and $|\frac{f(y)-f(x)}{y-x}| < \epsilon$, then we can find a finite disjoint subset of intervals of $\mathcal{V}$ that cover all but $\delta$ of $[a,b]$ for $\delta$ sufficiently small, so $|f(b)-f(a)| \le \epsilon|b-a| + \epsilon$.
\end{proof}

\begin{ex} The Cantor function (aka the Devil's Staircase, defined by writing a number in ternary, ignoring every digit after the first $1$, replacing every $2$ with a $1$, and interpreting the result in binary) is uniformly continuous, but not absolutely continuous, and has derivative $0$ almost everywhere.

This example leads to other pathologies: let $f : [0,1] \rightarrow [0,1]$ be the Cantor function, let $h(x) = f(x)+x : [0,1] \rightarrow [0,2]$, let $g = h^{-1} : [0,2] \rightarrow [0,1]$, let $C \subseteq [0,1]$ be the Cantor set, and let $D = [0,1]\setminus C$. Since $D$ is a union of open intervals whose lengths sum to $1$ and $f$ is constant on these intervals, $h(D)$ is measurable with measure $1$, so $h(C)$ is also measurable with measure $1$. Any measurable subset with positive measure contains a set which isn't measurable (usual argument with equivalence classes based on rationals works), so let $A \subseteq h(C)$ be such a nonmeasurable set and let $B = g(A)$. Then $B \subseteq C$, so $B$ is Lebesgue measurable (but not Borel measurable), so $\chi_B,g$ are both measurable but $\chi_B\circ g$ is not measurable. Additionally, although $g,h$ are continuous and strictly increasing, we have $A = g^{-1}(B) = h(B)$ is not measurable even though $B$ is.
\end{ex}

% TODO: figure out what is the deal with the Calderon-Zygmund decomposition and the rest of the stuff on https://en.wikipedia.org/wiki/Singular_integral_operators_of_convolution_type#Calder%C3%B3n-Zygmund_decomposition and figure out where to include it

A lot of the following is from \cite{gauge-integral}.

\begin{prop} Absolutely continuous functions map null sets to null sets and map measurable sets to measurable sets.
\end{prop}

\begin{prop} If $f_n$ is a sequence of equi-absolutely continuous functions (i.e. for each $\epsilon$, a single $\delta$ works for all of them), and $\lim_n f_n = f$ pointwise, then $f$ is absolutely continuous (and similarly for a sequence with uniformly bounded variation). In particular, if a sequence $g_n$ of absolutely continous functions has $\sum_n g_n$ convergent and the sum of the variations of the $g_n$s is finite, then $\sum_n g_n$ is absolutely continuous.
\end{prop}

\begin{prop} If $f$ has bounded variation on $[a,b]$, $V(x)$ is the variation of $f$ on $[a,x]$, and $f$ is continuous at $c$, then $V$ is also continuous at $c$. In particular, if $f$ is continuous and has bounded variation on $[a,b]$, and is absolutely continuous on $[a,c]$ for all $a<c<b$, then $f$ is absolutely continuous on $[a,b]$.
\end{prop}

\begin{prop} Suppose $f:[a,b] \rightarrow \mathbb{R}$ is bounded, and let $m_f(x) = \limsup_{t\rightarrow x} f(t)$. Then $m_f$ can be written as a pointwise limit of step functions which each exceed $f$, and the upper Riemann integral of $f$ is equal to the Lebesgue integral $\int m_f$. In particular, a bounded function on $[a,b]$ is Riemann integrable iff it is continuous a.e.
\end{prop}

\begin{prop} There is a perfect, nowhere dense subset of $[0,1]$ with positive Lebesgue measure.
\end{prop}
\begin{proof} The construction is the same as the Cantor set, but shrink each removed interval by a constant factor. Alternatively, consider the set of numbers whose base $5$ expansions contain no $2$s.
\end{proof}

\begin{prop} There is a function $f:[0,1] \rightarrow \mathbb{R}$ such that $f'$ exists and is bounded everywhere on $[0,1]$, but $f'$ is discontinuous on a set of positive measure and is therefore not Riemann integrable.
\end{prop}
\begin{proof} Let $E$ be a perfect, nowhere dense subset of $[0,1]$ with positive measure. The plan is to make $f$ equal to $0$ on $E$, and on each open interval $(a,b)$ in $[0,1]\setminus E$ to choose $f$ such that $|f(x)| \le |x-a|^2, |b-x|^2$, but such that $|f'(x)| = 1$ for points $x \in (a,b)$ arbitrarily close to $a$ and $b$. To construct such functions, start with the function $x \mapsto (x-a)^2\sin(1/(x-a))$ around $a$, and connect it to a similar function around $b$.
\end{proof}

\begin{prop} If $f_i$ are nondecreasing functions and $f = \sum_i f_i$ converges, then $f' = \sum_i f_i'$ a.e.
\end{prop}
\begin{proof} Let $g_k = \sum_{i>k} f_i$, then it's enough to prove that $\lim_k g_k' = 0$ a.e. To see this, pick a subsequence $k_i$ such that $\sum_i g_{k_i}$ converges at points $a,b$, and note that $\int_a^b \sum_i g_{k_i}' = \sum_i \int_a^b g_{k_i}' = \sum_i g_{k_i}(b) - g_{k_i}(a) < \infty$, so $\sum_i g_{k_i}'$ must converge a.e. on $[a,b]$, so $\lim_k g_k' = \lim_i g_{k_i}' = 0$ a.e. on $[a,b]$.
\end{proof}

\begin{prop} If $f : \mathbb{R} \rightarrow \mathbb{R}$ is measurable and $0 < \alpha < \beta$, then the function $x \mapsto \sup\{\frac{f(y)-f(x)}{y-x} \mid x+\alpha < y < x+\beta\}$ is measurable. In particular, all four derivates $D^+f, D^-f, D_+f, D_-f$ are measurable.
\end{prop}
\begin{proof} If $\sup\{\frac{f(y)-f(x)}{y-x} \mid x+\alpha < y < x+\beta\} > r$, then $x$ is contained in one of countably many sets which are preimages of open intervals under $f$, intersected with open sets that guarantee the $\sup$ is large so long as $f(x)$ is in the given range.
\end{proof}

\begin{prop} If $f : \mathbb{R} \rightarrow \mathbb{R}$ is any function, then $\overline{D}f : x \mapsto \limsup_{y\rightarrow x} \frac{f(y)-f(x)}{y-x}$ is measurable, as is the similarly defined $\underline{D}f$.
\end{prop}
\begin{proof} Let $r \in \mathbb{R}$, and for any $k,n$ let $E^k_n$ be the union of all intervals $[a,b]$ such that $b-a < \frac{1}{k}$ and $\frac{f(b)-f(a)}{b-a} > r+\frac{1}{n}$. Then $E^k_n$ is measurable since it is a union of closed intervals, and the set where $\overline{D}f > r$ is equal to $\cup_n \cap_k E^k_n$.
\end{proof}

\begin{ex} If $E \subseteq [0,1]$ is null, then there is a nondecreasing absolutely continuous function $f$ with $D_-f = D_+f = +\infty$ on $E$. To see this, for each $n$ find an open set $O_n$ containing $E$ with measure at most $\frac{1}{2^n}$, let $f_n$ be the integral of $\chi_{O_n}$, and let $f = \sum_n f_n$.
\end{ex}

\begin{ex} Let $E$ be a perfect nowhere dense set of positive measure in $[0,1]$, and let $f$ be the integral of $\chi_{[0,1]\setminus E}$. Then $f$ is strictly increasing and absolutely continuous, but $f' = 0$ on a set of positive measure. Additionally, by summing countably many dilated copies of the Cantor function, we can make a strictly increasing continuous function which has derivative $0$ almost everywhere.
\end{ex}

%\subsubsubsection{Gauge Integral}

%TODO: everything on https://en.wikipedia.org/wiki/Henstock%E2%80%93Kurzweil_integral - maybe see paper on role of continuum (saved as uncountable reverse math).

%\subsubsection{Bochner and Pettis integral}

% TODO: Pettis measurability theorem, definitions, basic properties...


\subsubsection{$L^p(X,\mu)$}

% TODO: fill in proofs for this section

\begin{defn} We say that a function is \emph{null} if it vanishes outside of a set of measure $0$.
\end{defn}

\begin{defn} For $p > 0$, the $p$-\emph{norm} of a measurable function $f : X \rightarrow \mathbb{C}$ (possibly undefined or infinite on a set of measure zero) with respect to the measure $\mu$ is defined by $\|f\|_p = (\int_X |f|^p\ d\mu)^{1/p}$ for $p < \infty$, and $\|f\|_\infty = \inf\{C \ge 0 \mid |f(x)| \le C\text{ a.e. }x \in X\}$. We let $\mathcal{L}^p(X,\mu)$ be the vector space of functions on $X$ with $\|f\|_p < \infty$, and we let $L^p$ be the quotient of $\mathcal{L}^p$ by the set of null functions.
\end{defn}

\begin{prop} If $f, g \in L^p$ then $f+g \in L^p$. If $0 < p \le 1$ then $d_p(f,g) = \|f-g\|_p^p$ defines a metric on $L^p$.
\end{prop}
\begin{proof} For $1 \le p < \infty$, we have $|f+g|^p \le 2^{p-1}(|f|^p + |g|^p)$ by convexity of $|\cdot |^p$, while for $0 < p \le 1$ we have $|f+g|^p \le 0^p + (|f|+|g|)^p \le |f|^p + |g|^p$ by concavity of $(\cdot )^p$ on $\mathbb{R}^+$.
\end{proof}

\begin{prop} If $f \in L^\infty \cap L^q$ for some $q < \infty$, then $\|f\|_\infty = \lim_{p \rightarrow \infty} \|f\|_p$.
\end{prop}

\begin{lem}[Young's Inequality] If $a,b,p,q \ge 0$ and $\frac{1}{p} + \frac{1}{q} = 1$, then $ab \le \frac{a^p}{p} + \frac{b^q}{q}$, with equality when $a^p = b^q$.
\end{lem}

\begin{thm}[H\"older]\label{Holder} If $\frac{1}{p} + \frac{1}{q} = 1$ and $f \in L^p, g \in L^q$, then $\|fg\|_1 \le \|f\|_p \|g\|_q$.

Conversely, if $p < \infty$ and $f \in L^p$, then $\|f\|_p = \max\{|\int_X fg\ d\mu| \text{ s.t. } \|g\|_q \le 1\}$, and the same holds for $p = \infty$ with the $\max$ replaced by a $\sup$ if every set of infinite measure contains a subset of finite nonzero measure.
\end{thm}
\begin{proof} We may assume without loss of generality that $\|f\|_p = \|g\|_q = 1$. Then $\int |fg|\ d\mu \le \int \frac{|f|^p}{p} + \frac{|g|^q}{q}\ d\mu = 1$. Without the assumption that $\|f\|_p = \|g\|_q = 1$, the argument goes as follows:
\[
\int |fg| = \|f\|_p\|g\|_q\int \frac{|f|}{\|f\|_p}\frac{|g|}{\|g\|_q} \le \|f\|_p\|g\|_q \int \frac{|f|^p}{p\|f\|_p^p} + \frac{|g|^q}{q\|g\|_q^q} = \|f\|_p\|g\|_q.
\]
For the converse, take $g = \frac{|f|^p}{f\|f\|_p^{p-1}}$.
\end{proof}

\begin{cor} If $\mu$ is $\sigma$-finite, then for $1 \le p,q \le \infty$ and $\frac{1}{p} + \frac{1}{q} = 1$, we have $f \in L^p$ iff there exists some $M$ such that $|\int fg| \le M\|g\|_q$ for all simple functions $g$.
\end{cor}
\begin{proof} Approximate $|f|$ from below by simple functions in $L^p$, and apply the converse to H\"older \ref{Holder}.
\end{proof}

\begin{thm}[Minkowski]\label{Minkowski} If $p \ge 1$, then $\|f+g\|_p \le \|f\|_p + \|g\|_p$, and if $1 < p < \infty$ we have equality iff $f = \lambda g$ with $\lambda \ge 0$ or $g = 0$ (a.e.).
\end{thm}
\begin{proof} By H\"older \ref{Holder}, for any $h \in L^q$ with $\frac{1}{p} + \frac{1}{q} = 1$, we have $\int |f+g|h \le \|f\|_p\|h\|_q + \|g\|_p\|h\|_q$, and taking $h = |f+g|^{p-1}$ gives the result (this $h$ has $\int |f+g|h = \|f+g\|_p\|h\|_q$).

For the equality case, note that by the equality case of Young's inequality in the proof of H\"older, we must have $\frac{|f|^p}{\|f\|_p^p} = \frac{|f+g|^{(p-1)q}}{\|f+g\|_p^{(p-1)q}} = \frac{|g|^p}{\|g\|_p^p}$ (a.e.).
\end{proof}

\begin{thm}[Riesz-Fischer for $L^p$]\label{Riesz-Fischer-Lp} $L^p$ is complete with respect to the $p$-norm for $0 < p \le \infty$.
\end{thm}
\begin{proof} It's enough to show that if $\sum_i \|u_i\|_p < \infty$ (or $\sum_i \|u_i\|_p^p < \infty$ in the case $0 < p < 1$) then $\sum_i u_i$ is the $L^p$-limit of its partial sums. This follows from the monotone convergence theorem (to show that $\sum_i |u_i|$ is in $L^p$), followed by the dominated convergence theorem to show that the tail sums converge to $0$ in the $p$-norm.
\end{proof}

\begin{cor} If $f_k$ converge in $L^p$ to $f$, then there is a subsequence $f_{k_i}$ that converge pointwise a.e. to $f$.
\end{cor}
\begin{proof} Choose any subsequence such that $\sum_i \|f_{k_{i+1}} - f_{k_i}\|_p < \infty$.
\end{proof}

\begin{prop} The integrable simple functions are dense in $L^p$ for every $0 < p < \infty$, and the simple functions are dense in $L^\infty$.
\end{prop}

\begin{thm}[Riesz Representation for $L^p$] The natural map $L^p \rightarrow L^{q*}$ is an isometric isomorphism if $1 < p < \infty$ and $\frac{1}{p} + \frac{1}{q} = 1$. If $\mu$ is $\sigma$-finite, then so is the map $L^\infty \rightarrow L^{1*}$.
\end{thm}
\begin{proof} By H\"older \ref{Holder}, we just need to check that $L^p \rightarrow L^{q*}$ is surjective. Let $I$ be a bounded linear functional on $L^{q*}$, we just need to construct an $f \in L^p$ such that $I(\chi_E) = \int f\chi_E$ for all sets $E$ with $\mu(E) < \infty$.

If $\mu$ is $\sigma$-finite, with the full space written as a disjoint union $\bigcup_i X_i$ with $\mu(X_i) < \infty$, then $\nu : E \mapsto \sum_i I(\chi_{E\cap X_i})$ defines a measure, and since $\mu(E) = 0 \implies \|\chi_E\|_q = 0 \implies I(\chi_E) = 0$, we have $\nu \ll \mu$. Then taking $f$ to be the Radon-Nikodym derivative $\frac{d\nu}{d\mu}$, we get $I(g) = \int fg$ for all integrable simple functions $g$, and since $I$ is a bounded functional on $L^q$ we see from the corollary to H\"older \ref{Holder} that $f \in L^p$.

For the general case, use the previous case to define functions $f_E \in L^p$ supported on $E$ for every $\sigma$-finite set $E$, such that $I(g) = \int f_Eg$ for $g \in L^q$ supported on $E$. For any $E \subseteq E'$ $\sigma$-finite, by uniqueness we have $f_E = f_{E'}|_E$ a.e., and $\|f_E\|_p \le \|f_{E'}\|_p \le \|I\|$. Choose a sequence $E_i$ of $\sigma$-finite sets with $\lim_i \|f_{E_i}\|_p = \sup_E \|f_E\|_p$, and let $X = \bigcup_i E_i$. Then $X$ is $\sigma$-finite and $\|f_X\|_p = \sup_E \|f_E\|_p$, so for any $\sigma$-finite $E$ we have $f_E$ supported on $E \cap X$ up to a set of measure $0$. For any $g \in L^q$, the support of $g$ is a $\sigma$-finite set $E$, so $I(g) = \int f_Eg = \int f_{E\cap X}g = \int f_Xg$. Thus we may take $f = f_X$.
\end{proof}

\begin{prop} If $0 < p \le q \le \infty$ and $\mu(X) < \infty$ then $\|f\|_p \le \mu(X)^{\frac{1}{p} - \frac{1}{q}}\|f\|_q$.
\end{prop}
\begin{proof} By raising both sides to the $p$th power and replacing $f$ with $|f|^p$ and $q$ with $\frac{q}{p}$, we see that it's enough to prove $\|f\|_1 \le \mu(X)^{1-\frac{1}{q}}\|f\|_q$ for $1 \le q \le \infty$. This follows from H\"older \ref{Holder} applied to the functions $1$ and $f$.
\end{proof}

\begin{prop} If $f \in L^p, g \in L^q$, $\alpha \in [0,1]$, and $\frac{1}{r} = \alpha \frac{1}{p} + (1-\alpha) \frac{1}{q}$, then $\||f|^{\alpha}|g|^{1-\alpha}\|_r \le \|f\|_p^{\alpha}\|g\|_q^{1-\alpha}$. In particular, if $f \in L^p\cap L^q$, then $f \in L^r$ for all $r \in [p,q]$.
\end{prop}
\begin{proof} Apply H\"older \ref{Holder} to $|f|^{\alpha r} \in L^{p/\alpha r}$ and $|g|^{(1-\alpha)r} \in L^{q/(1-\alpha)r}$.
\end{proof}

\begin{prop} If $X$ is metrizable and $\sigma$-finite and $\Sigma$ is the Borel $\sigma$-algebra, then $C(X) \cap L^p$ is dense in $L^p$.
\end{prop}

\begin{prop} If $\mu$ is a Radon measure on a locally compact Hausdorff space, then continuous functions with compact support are dense in $L^p$ for $0 < p < \infty$.
\end{prop}
\begin{proof} It's enough to approximate $\chi_E$ for every Borel set $E$ with $\mu(E) < \infty$. For such $E$ and for any $\epsilon > 0$, there is an open set $U$ and a compact set $K$ with $K \subseteq E \subseteq U$ with $\mu(U\setminus K) < \epsilon$. By locally compact Urysohn \ref{lch-urysohn}, there is a continuous function $f$ taking values in $[0,1]$ which is supported on a compact subset of $U$, with $f|_K = 1$.
\end{proof}

\begin{cor} Integrable step functions are dense in $L^p(\RR^n)$ for $0 < p < \infty$.
\end{cor}


\subsection{Banach spaces}

\begin{defn} If $V$ is a vector space (over $\RR$ or $\CC$), then $p:V \rightarrow [0,\infty)$ is a \emph{seminorm} if $p(0) = 0$, $p(cv) = |c|p(v)$ for $c$ a scalar and $v\in V$, and $p(v+w) \le p(v) + p(w)$ for $v,w \in V$. $p$ is a \emph{norm} if additionally $p(v) = 0 \iff v = 0$.
\end{defn}

The next bit is stolen from \href{https://math.blogoverflow.com/2014/06/25/zabreikos-lemma-and-four-fundamental-theorems-of-functional-analysis/}{this blogoverflow post}.

\begin{lem}[Zabreiko's Lemma] If $X$ is a Banach space and $p : X \rightarrow [0,\infty)$ is a seminorm such that for all absolutely convergent series $\sum_{n=1}^\infty x_n$ in $X$ we have $p(\sum_n x_n) \le \sum_n p(x_n)$, then $p$ is continuous, that is, $p(x) \ll \|x\|$.
\end{lem}
\begin{proof} Let $A_n = p^{-1}([0,n])$, then since $X = \cup_n \overline{A_n}$, there is some $n$ such that $\overline{A_n}$ has nonempty interior by the Baire category theorem. Since $\overline{A_n}$ is convex and symmetric, some open ball $B_R(0)$ around $0$ is contained in $\overline{A_n}$. We claim that $B_R(0) \subseteq A_n$ as well: if $\|x\| < R$, pick $0 < q < 1$ such that $\frac{\|x\|}{1-q} < R$, set $y = \frac{R}{\|x\|}x$, then since $y \in \overline{A_n}$ there exists $y_0 \in A_n$ with $\|y - y_0\| < qR$, and then inductively we find $y_0, y_1, ... \in A_n$ such that for each $k$, we have $\|y - \sum_{i<k} y_i\| < q^kR$: $y_k$ is taken to be a point in $A_n$ with $\|q^{-k}(y - \sum_{i<k} y_i) - y_k\| < qR$. Since $\|y_k\| < R + qR$ for each $k$, the sum $\sum_k q^k y_k = y$ is absolutely convergent, so by hypothesis $p(y) \le \sum_k q^kp(y_k) \le \frac{n}{1-q}$, so $p(x) \le \frac{\|x\|}{R}\frac{n}{1-q} < n$, so $x \in A_n$.
\end{proof}

\begin{thm}[Open Mapping Theorem] If $X,Y$ Banach spaces, $A:X\rightarrow Y$ surjective and continuous, then $A$ takes open sets to open sets.
\end{thm}
\begin{proof} For $y \in Y$, set $p(y) = \inf \{\|x\| \mid Ax = y\}$ in Zabreiko's Lemma.
\end{proof}

\begin{thm}[Bounded Inverse Theorem] If $X,Y$ Banach spaces, $A:X\rightarrow Y$ bijective and continuous, then $A^{-1}$ is also bounded.
\end{thm}

\begin{thm}[Closed Graph Theorem] If $X, Y$ Banach spaces, then $A:X\rightarrow Y$ is bounded iff the graph is closed in $X\times Y$.
\end{thm}
\begin{proof} For $x \in X$, set $p(x) = \|Ax\|$ in Zabreiko's Lemma.
\end{proof}

\begin{thm}[Uniform Boundedness Theorem/Banach Steinhaus] If $X$ is Banach, $Y$ a normed vector space, $F$ a set of continuous linear functions $T:X\rightarrow Y$. If $\forall x \in X\ \sup_{T\in F} \|T(x)\| < \infty$, then $\sup_{T \in F} \|T\| < \infty$.
\end{thm}
\begin{proof} Set $p(x) \in \sup_{T\in F} \|T(x)\|$ in Zabreiko's Lemma.
\end{proof}

\begin{cor} If a sequence of bounded operators from a Banach space to a normed space converges pointwise, then the pointwise limit is a bounded operator.
\end{cor}


\bibliographystyle{plain}
\bibliography{all}

\end{document}

